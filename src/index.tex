\documentclass[final]{elteikthesis}[2025/03/25]

\usepackage{abstract}
\usepackage[newfloat]{minted}
\usepackage{dirtree}

\setminted{fontsize=\footnotesize}

\removefromtoclist[float]{lol}

\title{Git Chord: Git tárolók meta-állapotainak hatékony kezelése}
\date{2025}

\author{Horváth Dávid}
\degree{Programtervező Informatikus BSc}

\supervisor{Dr. Cserép Máté}
\affiliation{egyetemi adjunktus}

\university{Eötvös Loránd Tudományegyetem}
\faculty{Informatikai Kar}
\department{Programozáselmélet és Szoftvertechnológiai\\ Tanszék}
\city{Budapest}
\logo{elte_cimer_szines}

\addbibresource{references.bib}

\begin{document}

\documentlang{hungarian}

\listoftodos
\cleardoublepage

\maketitle

\renewcommand{\abstractname}{Témabejelentő}

\begin{abstract}
Szakdolgozatomban bemutatom a Git Chord programot és az ahhoz készített néhány további segédeszközt.
A Git Chord egy POSIX shellben írt kiterjesztés a Git
verziókezelő rendszerhez, amely konfigurálható módon lehetővé teszi egy helyi tároló elemeinek
(branchek, tagek, stb.)
aktuális állapotából összeálló konstelláció mentését és visszaállítását,
illetve ezeknek a mentett meta-állapotoknak a böngészését, törlését, migrálását, másokkal való megosztását.

A Git Chord kényelmesebb megoldást kínál a korábbi meta-állapotok visszaállítására,
mint a Git által beépítetten támogatott reflog.
A reflog továbbá csak a helyi tárolón használható,
míg a Git Chord mentett meta-állapotai könnyen megoszthatók,
például migráció, deployment, archiválás vagy akár segítségkérés céljából.

A mentett meta-állapotok technikailag a tárolón belüli
speciális brancheken elhelyezett, teljesen reguláris commitok,
melyek többedik parent commitokként is hivatkoznak a meta-állapot által rögzített, hordozandó commitokra.
Így minden szükséges függőség garantáltan szállításra kerül a meta-állapot commitjával együtt,
ugyanakkor a mentések pehelysúlyúak, valójában nem történik költséges duplikáció.

A parancssori program mellett a szakdolgozati munka része egy webes GUI,
valamint az előbbire épülő kiegészítők néhány elterjedt
általános asztali integrált fejlesztői környezethez (például VS Code, Eclipse).
Az egyes kiegészítők a közös főképernyő mellett
az adott IDE sajátosságaihoz igazított egyedi működéseket is tartalmaznak (context menük, akciók, stb.).
A programnak a szakdolgozati védés keretében történő bemutatása során
nagy részben a webes felületre támaszkodom majd.

A fejlesztői-üzemeltetői hétköznapok során történő felhasználási eseteken túl
kitérek a programnak a verziókezelő rendszerek oktatásában való lehetséges alkalmazására.
Ennek alapgondolata, hogy a hallgatók által
a saját helyi tárolójukon végzett lokális módosítások eredménye
az értékelő oktatóval könnyen megosztható a Git Chord által mentett meta-állapotként.
\end{abstract}

\tableofcontents
\cleardoublepage

\chapter{Bevezetés}

\section{A modern verziókezelő rendszerek sajátosságai}

A verziókezelő rendszerek fő feladata,
hogy egy folyamatosan változó tartalomegyüttesnek (például egy szoftver forráskódjának)
az evolucióját megkönnyítsék, ennek a folyamatnak technikai keretet adjanak.
E tekintetben a legfontosabb megoldandó feladat minden bizonnyal az,
hogy a rendszer képes legyen eltárolni a tartalomegyüttes bizonyos jóváhagyott állapotainak sorát.
Ez a régi rendszereken még fájlonként történt,
a modern megközelítésben ezzel szemben a teljes tartalomegyüttes
egy-egy pillanatképe, snapshotja jelenti az atomi egységet.
További alapvető feladat, amelyet egy ilyen rendszernek biztosítania kell, a párhuzamos fejlesztés támogatása.
A régi rendszerekben még egy-egy fájl központilag történő lockolása volt hivatott biztosítani azt,
hogy több fejlesztő ne írja fölül egymás munkáját.
A modern rendszerekben a párhuzamos fejlesztés külön ágakon, azaz brancheken párhuzamosan történik.

Jelenleg a legelterjedtebb verziókezelő a nyílt forráskódú Git.

[a git mentális modelljéről és adattárolásáról, ez a továbbiakhoz szükséges lesz]

-------------------

párhuzamos fejlesztés támogatása,
azaz annak menedzselése, hogy a forráskód különböző részein is dolgozni lehessen anélkül,
hogy egymásra kellene várni.


Egy \textit{verziókezelő rendszer} feladata, hogy valamilyen adott tartalom (például programkód)
folyamatos fejlesztését megkönnyítse, annak evolúcióját támogató technikai keretet biztosítson.
A verziókezelt tartalom újabb és újabb verziói egy időbeli sorba rendeződnek,
a verziók és a köztük lévő különbségek visszakövethetők,
az egyes állapotok szükség esetén visszaállíthatók.
Emellett a fejlesztés el is ágazhat,
ekkor az újabb verziók párhuzamosan futó \textit{branch}eken kerülnek tárolásra.
Az elágazó fejlesztési történet lineáris láncolat helyett tehát egy aciklikus gráf. 
A modern elosztott verziókezelőkben a branchek kezelésére igen szofisztikált eszköztár áll rendelkezésre;
az olyan műveletek, mint új branchek létrehozása, meglévők egyesítése,
váltás a branchek között,
vagy akár a történet visszamenőleges manipulációja
általában egyszerű parancsok végrehajtásával elvégezhetők.

Jelenleg a legelterjedtebb verziókezelő a nyílt forráskódú \textit{Git}.
Filozófiájához hozzátartozik,
hogy a teljes verziókezelési folyamatot a helyi gépen kell elvégezni,
ezt lehet majd szinkronizálni más gépekre, például egy központi szerverre.
Mind a Git szervert, mind a lokális másolatot tárolónak (repository),
vagy röviden \textit{repó}nak nevezzük.
A Git elosztott rendszer,
használatához nincs feltétlenül szükség központi tárolóra.
Egy módosításcsomag publikálása két lépésből áll.
Elsőként a \textit{commit} művelettel létrehozzuk az új tárolt állapotot (snapshot);
a bevett terminológia szerint ezt is commitnak nevezik.
[TODO: a commit tartalma]
Alapértelmezetten az aktuális branch mentett állapota egyúttal az újonnan létrejött commitra ugrik.
Második lépésben a \textit{push} művelettel publikálhatjuk az adott branch (vagy branchek) állapotát egy szerverre.
A két lépés egymástól független,
és gyakran számos több commit létrehozása is megtörténik lokálisan,
mire sor kerülne a push műveletre.
A brancheken kívül \textit{tag}ek is létrehozhatók,
ezek hasonlóak a branchekhez,
a fő különbség, hogy véglegesek:
ha egyszer létrehoztunk egy adott commitra mutató taget,
az a továbbiakban immutábilisen erre a commitra fog mutatni
(ha mégis mozgatni szeretnénk, törölni kell, majd újra létrehozni).
Valójában kétféle tag létezik,
[TODO: a kétféle tag]
Általában (bár ez megkerülhető) a fejlesztők a saját gépükön
egyszerre tárolják a teljes verziótörténetet,
beleértve a számukra érdekes brancheket, gyakran az összeset.
Amikor a \textit{clone} művelettel lemásolunk, leklónozunk egy repót,
alapértelmezetten az összes branch és tag letöltésre kerül a teljes történettel együtt.
[vagyis sok van, itt jön be a nehézség; kezdők...]
[meta-verziókezelés, bár egy elágazó gráf, mindig van egy pillanatnyi meta-állapot]

\section{A Git verziókezelő rendszer tárolási modellje}

TODO

\section{Tárolók átfogó állapotának kezelése}

[a subversion stb. még egy nagy (idődimenzióval kiegészített) fájlrendszerben tárolta a brancheket is, valójában nincs beépített branchkezelés, az ágak egyszerűen külön könyvtárakban vannak, az elágaztatási módszertanok működtetését a könyvtárak között alkalmazható merge-elő, összehasonlító stb. eszközök segítik (megjegyzés: ezt a git is tudja)]

TODO

\section{A Git Chord megközelítése}

TODO

\section{Edge Case-ek a Git Chord-ban}

[persze eleve: a Git Chord teljesen a git saját szisztémájába illeszkedik]

submodule, git config módosításai, 

TODO

\cleardoublepage

\chapter{Felhasználói dokumentáció}

\section{A parancssori program használata}

\subsection{Bevezetés}

A Git Chord egy POSIX Shellben írt kiegészítő a Git rendszerhez.
Egy repository átfogó metaállapotának teljes vagy részleges snapshotjait képes kezelni.
Ezek a snapshotok speciális (alapértelmezetten \texttt{chord/} prefixű) brancheken kerülnek tárolásra,
és kizárólag a Git rendszer natív képességeire építenek.

Egy-egy snapshot tárolja alapértelmezetten a branchek és tagek állapotát tárolja.
Ez a működés konfigurációval felülírható, finomhangolható,
illetve lehetőség van a staging area és a munkakönyvtár állapotának tárolására is.
Egyazon repository által kezelt több munkakönyvtár (\texttt{git-worktree}) tárolása jelenleg nem támogatott.
Lehetőség van a snapshotok létrehozására, böngészésére, törlésére, megosztására és letöltésére is.

\subsection{Telepítés és alapvető használat}

A Git Chord bármilyen POSIX-kompatibilis rendszeren futtatható.
Ezen felül egyetlen függősége maga a Git rendszer,
melynek legalább 2.0.0 verziója szükséges a teljeskörű használathoz.

A \texttt{bin} könyvtárban található \texttt{git-chord} szkriptfájl önmagában vagy a \texttt{git} parancs alparancsként is használható.
Az alparancsként való használathoz elengendő, ha a szkript változatlan névvel a végrehajtható fájlok keresési könyvtárainak egyikében található (POSIX rendszereken ezt a \texttt{PATH} környezeti változó reprezentálja).

Interaktív terminálban Bash héj esetén automatikus parancskiegészítést is használhatunk,
miután a \texttt{completion} könyvtárban található \texttt{git-chord-completion.bashrc} szkript be lett emelve (a `source` vagy `.` paranccsal).
Ajánlott ezt a beemelést a \texttt{bashrc} mechanizmusba helyezni,
hogy már a bejelentkezéskor lefusson.

Mindezek az inicializáló lépések az \texttt{install} könyvtárban található szriptek valamelyikével is elvégezhetők.
Az itt található \texttt{netinstall-user.sh} szkript a letöltéstől a teljes telepítésig elvégzi a folyamatot.
Hálózati elérés és a \texttt{curl} eszköz megléte esetén így a teljes telepítés elvégezhető egyetlen paranccsal:

\begin{minted}[breaklines]{shell}
curl -s https://raw.githubusercontent.com/davidsusu/git-chord/main/install/netinstall-user.sh | sh
\end{minted}

Sikeres telepítés után a programot önállóan (\texttt{git-chord})
vagy a \texttt{git} alparancsaként (\texttt{git chord}) is használhatjuk.
A plusz argumentumok kezelése és a működés a két esetben ugyanaz.
A továbbiakban mindig az alparancsos stílust fogom használni.

Használatba vétel előtt érdemes lehet ellenőrizni a telepített verziót:

\begin{minted}{shell}
git chord version
\end{minted}

Vagy röviden:

\begin{minted}{shell}
git chord -v
\end{minted}

Az első esetben a \verb|git-chord| program \texttt{version} alparancsát,
a második esetben a \verb|-v| POSIX kapcsolót használtuk.

További alapvető alparancs a \verb|help|, mely egy általános leírást ír ki,
magába foglalva az elérhető alparancsokat, kapcsolókat, paramétereket és egyéb információkat.
Ugyanerre a \verb|-h| kapcsoló is használható.

Az egyes alparancsokhoz saját leírás is elérhető,
ha az alparancs után további paraméterként megadjuk a \verb|help| szót
(vagy a \verb|-h| kapcsolót).
Például:

\begin{minted}{shell}
git chord snapshot help
\end{minted}

Részlet a kimenetből:

\begin{minted}{text}
git chord snapshot

Usage:
  git chord snapshot [<name>]

  git chord snapshot - <branch>

Example:
  git chord snapshot my-snapshots

Description:
  Creates and saves a snapshot of the entire state of [...]
\end{minted}

Ha a terminál támogatja az ANSI formázási escape szekvenciákat,
akkor a kimenet alapértelmezetten formázva jelenik meg.
Ez letiltható a \verb|--no-color| opció használatával.
(Általában a logikai paraméterek esetében a \verb|--no-| prefix tiltást,
a sima \verb|--| prefix pedig ellenkezőleg, kikényszerítést jelent.)
További kimenet-opció a \verb|--markdown|, mely a Markdown formátumú kimenetet kapcsolja be (illetve a \verb|--no-markdown| letiltja).
Egyes parancsok esetében lehetőség van bővebb kimenet bekapcsolására a \verb|--verbose|,
illetve tiltására a \verb|--no-verbose| opció használatával.

A \verb|--defaults| logikai opció bármely parancs esetében használható,
ekkor a programba beépített alapértelmezett konfigurációs értékek töltődnek be.
Számos parancsnál használható továbbá az \verb|--all| logikai opció,
pontos jelentése azonban specifikus
(a parancs által célzott objektumtípusból az összeset fogja használni).

A műveletet végrehajtó parancsok esetén lehetőség van a várható működés szimulálására
a művelet tényleges végrehajtása nélkül.
Ehhez a \verb|--dryrun| kapcsolót kell megadni.

A \texttt{git chord} parancsok háromféle státuszkóddal léphetnek ki:

\begin{quote}
\begin{description}
    \item[0:] Normál futtatáskor, siker esetén.
    \item[1:] A műveletet meghiúsító hiba esetén.
    \item[2:] Szimulált működés esetén (\verb|--dryrun|), ha nincs hiba.
\end{description}
\end{quote}

\subsection{Konfiguráció}

A Git Chord működése konfigurálható értékek megadásával szabható testre.
Minden ilyen értékhez tartozik egy konfigurációs kulcs.
A kulcs pontokkal választott szavakból áll, például: \verb|trackers.name| .

A konfiguráció több szinten kezelhető, melyek kulcsonként rendre felülírják egymást.
A szintek a következők:

\begin{description}
    \item[Alapértelmezések:]
        Ezek a programba beépített kezdőértékek.
        Ha valamely konfigurációs kulcs sehol máshol nincs definiálva, az itt meghatározott érték lesz használva.
        Példa: \verb|trackers.name main| .
    \item[Git config:]
        A Git rendszerben tárolt konfigurációban \verb|chord.| prefixszű kulcsokkal lehet konfigurációs értékeket definiálni a Git Chordhoz.
        A Git saját konfigurációs mechanizmusa maga is többszintű,
        az aktuálisan érvényes értékek lesznek használva.
        Példa: \verb|chord.trackers.name=master| .
    \item[Parancssori opciók:]
        A konfiguráció egy-egy parancsnál is megadható.
        Az opciókat \verb|--| prefix vezeti be, ezt követi a kulcs neve.
        Az ajánlott konvenció itt a mínuszjelek használata a pont karakter helyett,
        a normalizáció során ugyanis az előbbi az utóbbira cserélődik.
        Nem logikai értéket váró kulcs esetén az értéket a név után új argumentumként (vagy pedig egyenlőségjellel választva) kell megadni.
        A logikai értéket váró kulcsok esetében alapértelmezetten kényszerítés történik,
        a \verb|--| helyett a \verb|--no-| prefixet használva pedig tiltás.
        Ismétlődés esetén a későbbi értékadás felülírja a korábbit.
        Példa: \verb|--trackers-name mybranch| .
\end{description}

A parancssorban néhány speciális opciót is használhatunk, melyek egyszerre több konfigurációs kulcshoz rendelnek értéket:

\begin{description}
    \item[\texttt{{-}{-}defaults}:]
        Nem vár értéket (logikai típusú).
        Kényszeríti az alapértelmezett értékek használatát.
        A később következő parancssori opciók viszont felülírják az alapértelmezést.
    \item[\texttt{{-}{-}fullstore}:]
        Nem vár értéket (logikai típusú).
        Az összes támogatott objektum tárolását engedélyezi, azaz teljeskörű mentést állít be.
    \item[\texttt{{-}{-}fullstore}:]
        Nem vár értéket (logikai típusú).
        Az összes támogatott objektum betöltését engedélyezi, azaz teljeskörű visszaállítást állít be.
    \item[\texttt{{-}{-}profile}:]
        Értékként egy profil nevét várja.
        Betölti a profilban definiált konfigurációs beállításokat.
        Bővebben lásd: \ref{subsec:profiles}.
\end{description}

Támogatott továbbá a speciális \verb|--| argumentum, mely után opciókat már nem lehet megadni, csak egyszerű argumentumokat (például további alparancs, érték, stb.).

A \verb|git chord config --default| paranccsal kilistázhatjuk az alapértelmezett konfigurációt
(ez akkor is működik, ha nem egy repository könytára alatt vagyunk).
A \verb|--default| opció nélkül használt \verb|git chord config| parancsokkal kezelhetjük a Git Chordnak a Git saját mechanizmusával tárolt konfigurációit
(ez csak a repository könytára alatt működik).
Ekkor a következő lehetőségek elérhetők:

\begin{description}
    \item[\texttt{git chord config} (vagy \texttt{git chord config list}):]
        Listázza az aktuálisan érvényes konfiguráció kulcs-érték párjait.
    \item[\texttt{git chord config get <key>}:]
        Kiírja a megadott \verb|<key>| kulcshoz rendelt értéket.
    \item[\texttt{git chord config set <key> <value>}:]
        A megadott \verb|<key>| kulcshoz beállítja a megadott \verb|<value>| értéket.
    \item[\texttt{git chord config reset <key>}:]
        Törli a \verb|<key>| kulcshoz rendelt értéket (azaz megszünteti a felülírást).
\end{description}

Vegyük például a következő parancsot:

\begin{minted}{shell}
git chord config set trackers.name mybranch
\end{minted}

Ez a \verb|trackers.name| kulcshoz a \verb|mybranch| értéket fogja rendelni az aktuális repositoryban.
Ehhez a repositoryhoz tartozó Git által kezelt konfigurációban a \verb|chord.trackers.name| kulcsot fogja használni.
Vagyis a fenti parancs megfelel ennek:

\begin{minted}{shell}
git config set chord.trackers.name mybranch
\end{minted}

\subsection{Profilok használata} \label{subsec:profiles}

A profilok konfigurációs felülírások egy halmazát nevesítik.
Megadásuk úgy történik, hogy az adott kulcsot kiegészítjük a \verb|profile.<name>.|
(tehát a Git konfigurációjában a \verb|chord.profile.<name>.|) prefixszel,
ahol a \verb|<name>| helyére kerül a profil neve.
Példa:

\begin{minted}{shell}
git chord config set profile.lorem.trackers.prefix lorem/
git chord config set profile.lorem.trackers.name ipsum
git chord config set profile.lorem.lightweighttags.store.enabled true
\end{minted}

Ezzel létrejön a \verb|lorem| nevű profil, mely három felülírást tartalmaz.
A profil betöltéséhez a \verb|--profile| opciót használhatjuk,
melynek paramétere a profil neve, ez esetben tehát \verb|lorem|.
Például:

\begin{minted}{shell}
git chord snapshot --profile lorem
\end{minted}{shell}

A profil meghatározása után megadott egyéb opciók a sorrendnek megfelelően
felülírják a profilban megadott értékeket
(tehát minden esetben az adott kulcsot érintő legutolsó opció nyer).
Egy parancson belül akár több profilt is betölthetünk,
ezek szintén a felsorolás rendjében fogják felülírni egymás (és az egyéb értékadások) hatását.

\subsection{Snapshotok létrehozása, visszaállítása és törlése}

A repó metaállapotáról a \verb|git chord snapshot| alparanccsal lehet snapshotot készíteni.
Például:

\begin{minted}{shell}
git chord snapshot
\end{minted}

A fenti parancs létrehoz egy snapshotot a repository aktuális metaállapotáról.
Ehhez a beállított speciális branchre (alapértelmezetten \verb|chord/main|) egy új commitba fog kerül,
ebben szerepel majd \verb|state.yaml| néven a snapshot YAML formátumú leírása.
Egy ilyen leírás például így nézhet ki (egyszerű esetben):

\begin{minted}{yaml}
timestamp: '2025-06-01T14:14:54Z'
branches:
    'main': '233f62b1bfbb0cbc401e1f8e497985c8cd614b7c'
annotatedTags: {}
head:
    pointingTo: 'main'
    ref: 'main'
    commitId: '233f62b1bfbb0cbc401e1f8e497985c8cd614b7c'
\end{minted}

A hivatkozott commitok (valamint a hivatkozott tree objektumokhoz automatikusan létrehozott commitok)
bekerülnek a snapshot commit szülő-commitjai közé.
Ez biztosítja (a Git hivatkozási szisztémáján keresztül),
hogy pusztán ennek a commitnak a hordozásával annak minden függősége is átadódik.

A speciális branch a \verb|trackers.prefix| és \verb|trackers.name| kulcsú konfigurációs értékek összefűzésével kerül meghatározásra.
A \verb|trackers.prefix| módosítása csak körültekintés mellett ajánlott,
ugyanis egyúttal ez az opció határozza meg,
hogy a Git Chord mely brancheket tekintse speciálisnak
(azaz hagyja ki mindenképpen a mentésből).

A \verb|--trackers-name| parancssori opció helyett a nevet egyszerűen plusz argumentumként is megadhatjuk:

\begin{minted}{shell}
git chord snapshot hello
\end{minted}

Ha pedig plusz argumentumként egyszerűen egy mínuszjelet írunk,
egy további paraméterként egy teljes branchnevet adhatjuk meg (csak kivételes esetekben ajánlott):

\begin{minted}{shell}
git chord snapshot - lorem/ipsum
\end{minted}

Hogy pontosan mi kerül mentésre, azt a következő konfigurációs értékek befolyásolják közvetlenül:

\begin{table}[H]
\captionsetup{justification=justified, singlelinecheck=false}
\begin{center}
\begin{tabular}{|l|l|>{\raggedright\arraybackslash\nohyphenation}p{5cm}|} 
 \hline
 \multicolumn{1}{|c}{\textbf{Kulcs}} &
 \multicolumn{1}{|c}{\textbf{Alapérték}} &
 \multicolumn{1}{|c|}{\textbf{Leírás}} \\
 \hline\hline
 
 \texttt{branches.store.enabled} & \texttt{true} & Branchek mentésének engedélyezése \vspace{0.5em} \\
 \texttt{branches.store.regex} & \texttt{.*} & Minta a mentendő branchek szűkítéséhez \vspace{0.5em} \\
 \texttt{annotatedtags.store.enabled} & \texttt{true} & Annotált tagek mentésének engedélyezése \vspace{0.5em} \\
 \texttt{annotatedtags.store.regex} & \texttt{.*} & Minta a mentendő annotált tagek szűkítéséhez \vspace{0.5em} \\
 \texttt{lightweighttags.store.enabled} & \texttt{false} & Lightweight tagek mentésének engedélyezése \vspace{0.5em} \\
 \texttt{lightweighttags.store.regex} & \texttt{.*} & Minta a mentendő lightweight tagek szűkítéséhez \vspace{0.5em} \\
 \texttt{head.store.enabled} & \texttt{true} & HEAD mutató mentésének engedélyezése \vspace{0.5em} \\
 \texttt{stagingarea.store.enabled} & \texttt{false} & A staging area mentésének engedélyezése \vspace{0.5em} \\
 \texttt{workingtree.store.enabled} & \texttt{false} & A munkakönyvtár állapota mentésének engedélyezése \vspace{0.5em} \\
 \texttt{fullstore} & \texttt{false} & Minden objektum mentésének kényszerítése \\
 
 \hline

\end{tabular}
\end{center}
\caption{A tárolást érintő konfigurációs kulcsok}}
\end{table}

Láthatjuk, hogy alapértelmezetten minden branch mentésre kerül,
és ez finomítható a szűrést megadó reguláris kifejezés módosításával.
Lényeges, hogy az annotált és lightweight tagek kezelése egymástól független.

A \verb|fullstore| rendesen csak ad hoc parancssori opcióként használatos,
és felülírja a táblázatban fölötte szereplő összes értéket a már ismertett módon.
A logikai értékeket mind \verb|true|-ra,
a \verb|.regex| végződésű kulcsok étékeit pedig \verb|.*|-ra állítja.

A következő parancs például pluszban menteni fogja az \verb|a| betűvel kezdődő lightweight tageket,
valamint a staging area állapotát:

\begin{minted}{shell}
git chord snapshot --lightweighttags-store-enabled true --lightweighttags-store-regex 'a.*' --stagingarea-store-enabled
\end{minted}

(Megjegyzés: az \verb|--all| opció nem helyettesíti a \verb|--fullstore| opciót.
Az \verb|--all| akkor használható, amikor egyértelmű a parancs által célzott objektumtípus,
abból összes kerül majd kiválasztásra
(például listázáskor az összes branch).)

Tehát ha hirtelen egy teljes mentést szeretnénk készíteni a konfigurácó állapotától függetlenül,
a következő paranccsal egyszerűen megtehetjük:

\begin{minted}{shell}
git chord snapshot --fullstore
\end{minted}

Vagy egy külön branchre mentve:

\begin{minted}{shell}
git chord snapshot --fullstore full
\end{minted}

A mentett snapshotok visszaállítása az \verb|apply| alparanccsal történik.
Kiadhatjuk például egyszerűen az alábbi parancsot:

\begin{minted}{shell}
git chord apply
\end{minted}

Ekkor szintén a \verb|trackers.prefix| és \verb|trackers.name| kulcsú konfigurációs értékek összefűzésével
kerül meghatározásra a speciális branch,
és ennek a branchnek az utolsó snapshot commitja kerül visszaállításra.
Ha nincs ilyen, akkor a parancs hibával leáll.

A \verb|snapshot| alparancshoz hasonlóan itt is használhatjuk a plusz argumentot a \verb|--trackers-name| helyett.
Például:

\begin{minted}{shell}
git chord apply hello
\end{minted}

Ez alapértelmezetten a \verb|chord/hello| branchen található utolsó snapshotot tölti be.
Ha nem a legutolsó snapshotot szeretnénk visszaállítani, akár a név megadása után, akár anélkül,
használhatjuk a sorszám-kapcsolót, azaz egy plusz argumentumot, mely egy mínuszjellel prefixelt nem-negatív egész szám
(a \verb|-0| jelenti a legutolsót, az \verb|-1| az utolsó előttit, és így tovább).
Például:

\begin{minted}{shell}
git chord apply hello -3
\end{minted}

Itt is használhatjuk a mínuszjel argumentumot, utána egy tetszőleges karakterlánc megadható,
melyet a Git rendszer commitként tud azonosítani.
Az előbbi parancsnak például megfelel a következő (feltéve, hogy a \verb|trackers.prefix| alapértelmezésre van állítva):

\begin{minted}{shell}
git chord apply - chord:hello~3
\end{minted}

\begin{table}[H]
\captionsetup{justification=justified, singlelinecheck=false}
\begin{center}
\begin{tabular}{|l|l|>{\raggedright\arraybackslash\nohyphenation}p{5cm}|} 
 \hline
 \multicolumn{1}{|c}{\textbf{Kulcs}} &
 \multicolumn{1}{|c}{\textbf{Alapérték}} &
 \multicolumn{1}{|c|}{\textbf{Leírás}} \\
 \hline\hline
 
 \texttt{branches.apply.enabled} & \texttt{true} & Branchek visszaállításának engedélyezése \\
 \texttt{branches.apply.regex} & \texttt{.*} & Minta a visszaállítandó branchek szűkítéséhez \\
 \texttt{branches.apply.allowremove} & \texttt{true} & Branchek törlésének engedélyezése \\
 \texttt{branches.apply.allowadd} & \texttt{true} & Branchek létrehozásának engedélyezése \\
 \texttt{annotatedtags.apply.enabled} & \texttt{true} & Annotált tagek visszaállításának engedélyezése \\
 \texttt{annotatedtags.apply.regex} & \texttt{.*} & Minta a visszaállítandó annotált tagek szűkítéséhez \\
 \texttt{annotatedtags.apply.allowremove} & \texttt{true} & Annotált tagek törlésének engedélyezése \\
 \texttt{lightweighttags.apply.enabled} & \texttt{false} & Lightweight tagek visszaállításának engedélyezése \\
 \texttt{lightweighttags.apply.regex} & \texttt{.*} & Minta a visszaállítandó lightweight tagek szűkítéséhez \\
 \texttt{lightweighttags.apply.allowremove} & \texttt{false} & Lightweight tagek törlésének engedélyezése \\
 \texttt{head.apply.enabled} & \texttt{true} & HEAD mutató mozgatásának engedélyezése \\
 \texttt{head.apply.sticktocommit} & \texttt{false} & A HEAD mutató mindenképpen a commithoz ragaszkodjon \\
 \texttt{stagingarea.followhead} & \texttt{true} & A staging area alapból a HEAD-et kövesse \\
 \texttt{stagingarea.apply.enabled} & \texttt{false} & A staging area állapotának visszaállításának engedélyezése \\
 \texttt{workingtree.followstagingarea} & \texttt{true} & A munkakönyvtár állapota alapból a staging area-t kövesse \\
 \texttt{workingtree.apply.enabled} & \texttt{false} & A munkakönyvtár állapotának visszaállításának engedélyezése \\
 \texttt{fullapply} & \texttt{false} & Minden objektum visszaállításának kényszerítése \\

 \hline

\end{tabular}
\end{center}
\caption{A tárolást érintő konfigurációs kulcsok}}
\end{table}

[tagek esetében a mozgatás a törlés és létrehozás együttesét jelenti, tehát igyényli az <...>.allowremove true-ra állítását]

[head.apply.sticktocommit]

...

% apply # Applies a previously saved snapshot to the repository
% delete # Deletes a previously saved snapshot

\subsection{Snapshotok böngészése}

% show # Shows full data of a previously saved snapshot
% list # Lists previously saved snapshots
% diff # Shows diff between a previously saved snapshot and the current state

% state # Shows the current state based on the configuration

\subsection{Snapshotok megosztása}

TODO

% push # Pushes the default chord branch (or using (`--all`): all chord branches)
% pull # Pulls the default chord branch (or using (`--all`): all chord branches)

\subsection{Egyéb parancsok}

TODO

% spec commands options # Gets special information mainly for machine processing

\section{A webes felület használata}

TODO

\cleardoublepage

\chapter{Fejlesztői dokumentáció}

\section{A parancssori \texttt{git-chord} program}

\subsection{A projekt forrásrepójának szerkezete}

Repó felépítése:

\dirtree{%
.1 git-chord/.
.2 bin/.
.3 git-chord.
.2 completion/.
.3 git-chord-completion.bashrc.
.2 install/.
.3 install-*.sh.
.2 maintain/.
.3 *.sh.
.2 packaging/.
.3 ,,,.
.2 test/.
.3 ,,,.
.2 LICENSE.
.2 README.md.
.2 regen.sh.
}

Használt alparancsok:

git add
git branch
git cat-file
git checkout
git commit-tree
git config
git for-each-ref
git hash-object
git log
git mktree
git push
git read-tree
git rebase
git remote
git reset
git rev-parse
git show
git show-ref
git stash
git symbolic-ref
git tag
git update-ref
git write-tree

\begin{table}[H]
\captionsetup{justification=justified, singlelinecheck=false, margin=0.7cm}
\begin{center}
\begin{tabular}{|l||l|l||} 
 \hline
 ~ & \multicolumn{1}{c|}{\textbf{XXX}} & \multicolumn{1}{c|}{\textbf{YYY}} \\
 \hline\hline
 
 \textbf{asdf} & lorem ipsum & lorem ipsum \\
 \textbf{fdsa} & lorem ipsum & lorem ipsum \\
 \hline

\end{tabular}
\end{center}
\caption{Lorem Ipsum}}
\end{table}

\subsection{A fő shell szkript felépítése}

POSIX-kompatibilis

Főszkript felépítése:

\begin{itemize}
    \item header
    \item definíciók
    \item segédfüggvények (string manipulation, arguments handling, YAML, file system, miscellaneous)
    \item beállítások, opciók és paraméterek feldolgozása, alparancs megállapítása
    \item formázók beállítása a kimenet-beállítások alapján
    \item az opcióktól is függő függvények
    \item (teljesen vagy feltételesen) repófüggetlen alparancsok kiszolgálása (help, version, default-kezelés stb.)
    \item git repó ellenőrzése
    \item repófüggő alparancsok kiszolgálása
\end{itemize}

A POSIX nem tartalmazza az \verb|mktemp| (és így \verb|mktemp -d|) stb. parancsokat, ezért:

\begin{listing}[H]
\begin{minted}{shell}
createTmpDir() (
    sysTmpDir="$TMPDIR"
    if [ -z "$sysTmpDir" ]; then
        sysTmpDir="/tmp"
    fi
    pathPrefix="${sysTmpDir}/git-chord.$( date '+%N' )"
    path="$pathPrefix"
    i='1'
    while [ -e "$path" ]; do
        path="${pathPrefix}."
        i=$(( i + 1 ))
    done
    mkdir -p "$path"
    printf '%s\n' "$path"
)
\end{minted}
\caption{Átmeneti könyvtár készítése, POSIX-kompatibilis}
\end{listing}

először:

\begin{listing}[H]
\begin{minted}{shell}
tempYamlFile1="$( mktemp )" # FIXME: mktemp is not POSIX
printf '%s\n' "$stateYaml1" > "$tempYamlFile1"
tempYamlFile2="$( mktemp )" # FIXME: mktemp is not POSIX
printf '%s\n' "$stateYaml2" > "$tempYamlFile2"
if checkTruthy "$configValue_color"; then
    diff -u -U3 --color "$tempYamlFile1" "$tempYamlFile2"
else
    diff -u -U3 "$tempYamlFile1" "$tempYamlFile2"
fi
rm "$tempYamlFile1"
rm "$tempYamlFile2"
\end{minted}
\caption{Átmeneti fájlok kezelése előtte}
\end{listing}

majd:

\begin{listing}[H]
\begin{minted}{shell}
tempDir="$( createTmpDir )"
tempYamlFile1="${tempDir}/state1.yaml"
printf '%s\n' "$stateYaml1" > "$tempYamlFile1"
tempYamlFile2="${tempDir}/state2.yaml"
printf '%s\n' "$stateYaml2" > "$tempYamlFile2"
if checkTruthy "$configValue_color"; then
    diff -u -U3 --color "$tempYamlFile1" "$tempYamlFile2"
else
    diff -u -U3 "$tempYamlFile1" "$tempYamlFile2"
fi
rm -R "$tempDir"
\end{minted}
\caption{Átmeneti fájlok kezelése utána}
\end{listing}

...

\begin{listing}[H]
\begin{minted}{shell}
while IFS= read -r configLine; do
    configValue="$( printf '%s\n' "$configLine" | cut -d ' ' -f 2- )"
    outVariableName="configValue_$( printf '%s\n' "$configLine" | cut -d ' ' -f 1 | tr '.' '_' )"
    eval "${outVariableName}='$( escapeValue "$configValue")'"
done <<EOF
$( printf '%s\n' "$config" )
EOF
\end{minted}
\caption{Eredeti eval loop}
\end{listing}

Javított:

\begin{listing}[H]
\begin{minted}{shell}
eval "$(
    printf '%s\n' "$config" | sed -E ':loop; s/^(\w+)\./\1_/; t loop' | while IFS=' ' read -r outVariableName configValue; do
        printf "configValue_%s='" "$outVariableName"
        escapeValue "$configValue"
        printf "'\n"
    done
)"
\end{minted}
\caption{Optimalizált eval loop}
\end{listing}

\subsection{TODO}

TODO

\section{A Git Chord webes felületének felépítése}

TODO

\cleardoublepage

\chapter{Összegzés és alkalmazások}

Hello.

\cleardoublepage

\appendix

\chapter{Teszteredmények}

Hello.

\cleardoublepage

\phantomsection
\addcontentsline{toc}{chapter}{\biblabel}
\printbibliography[title=\biblabel]
\cleardoublepage

\phantomsection
\addcontentsline{toc}{chapter}{\lstfigurelabel}
\listoffigures
\cleardoublepage

\phantomsection
\addcontentsline{toc}{chapter}{\lsttablelabel}
\listoftables
\cleardoublepage

\phantomsection
\addcontentsline{toc}{chapter}{\lstalgorithmlabel}
\listofalgorithms
\cleardoublepage

\phantomsection
\addcontentsline{toc}{chapter}{\lstcodelabel}
\lstlistoflistings
\cleardoublepage

\end{document}

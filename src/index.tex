\documentclass[
    %final,
]{elteikthesis}[2025/03/25]

\usepackage{abstract}
\usepackage[newfloat]{minted}
\usepackage{fontawesome5}
\usepackage{todonotes}

\setminted{fontsize=\footnotesize}

\removefromtoclist[float]{lol}

\title{Git Chord: Git tárolók meta-állapotainak hatékony kezelése}
\date{2025}

\author{Horváth Dávid}
\degree{Programtervező Informatikus BSc}

\supervisor{Dr. Cserép Máté}
\affiliation{egyetemi adjunktus}

\university{Eötvös Loránd Tudományegyetem}
\faculty{Informatikai Kar}
\department{Programozáselmélet és Szoftvertechnológiai\\ Tanszék}
\city{Budapest}
\logo{elte_cimer_szines}

\addbibresource{references.bib}

\newcommand{\todoref}[1]{\todo[inline, noinlinepar, color=red, textcolor=white, inlinewidth=0.6cm, caption={#1}]{\large \textbf{×}}}

\begin{document}

\documentlang{hungarian}

\listoftodos
\cleardoublepage

\maketitle
\cleardoublepage

\renewcommand{\abstractname}{Témabejelentő}

\begin{abstract}
Szakdolgozatomban bemutatom a Git Chord programot és az ahhoz készített néhány további segédeszközt.
A Git Chord egy POSIX shellben írt kiterjesztés a Git
verziókezelő rendszerhez, amely konfigurálható módon lehetővé teszi egy helyi tároló elemeinek
(branchek, tagek, stb.)
aktuális állapotából összeálló konstelláció mentését és visszaállítását,
illetve ezeknek a mentett meta-állapotoknak a böngészését, törlését, migrálását, másokkal való megosztását.

A Git Chord kényelmesebb megoldást kínál a korábbi meta-állapotok visszaállítására,
mint a Git által beépítetten támogatott reflog.
A reflog továbbá csak a helyi tárolón használható,
míg a Git Chord mentett meta-állapotai könnyen megoszthatók,
például migráció, deployment, archiválás vagy akár segítségkérés céljából.

A mentett meta-állapotok technikailag a tárolón belüli
speciális brancheken elhelyezett, teljesen reguláris commitok,
melyek többedik parent commitokként is hivatkoznak a meta-állapot által rögzített, hordozandó commitokra.
Így minden szükséges függőség garantáltan szállításra kerül a meta-állapot commitjával együtt,
ugyanakkor a mentések pehelysúlyúak, valójában nem történik költséges duplikáció.

A parancssori program mellett a szakdolgozati munka része egy webes GUI,
valamint az előbbire épülő kiegészítők néhány elterjedt
általános asztali integrált fejlesztői környezethez (például VS Code, Eclipse).
Az egyes kiegészítők a közös főképernyő mellett
az adott IDE sajátosságaihoz igazított egyedi működéseket is tartalmaznak (context menük, akciók, stb.).
A programnak a szakdolgozati védés keretében történő bemutatása során
nagy részben a webes felületre támaszkodom majd.

A fejlesztői-üzemeltetői hétköznapok során történő felhasználási eseteken túl
kitérek a programnak a verziókezelő rendszerek oktatásában való lehetséges alkalmazására.
Ennek alapgondolata, hogy a hallgatók által
a saját helyi tárolójukon végzett lokális módosítások eredménye
az értékelő oktatóval könnyen megosztható a Git Chord által mentett meta-állapotként.
\end{abstract}

\tableofcontents
\cleardoublepage

\chapter{Bevezetés}

\section{A modern verziókezelő rendszerek sajátosságai}

A verziókezelő rendszerek fő feladata,
hogy egy folyamatosan változó tartalomegyüttesnek (például egy szoftver forráskódjának)
az evolúcióját megkönnyítsék, ennek a folyamatnak technikai keretet adjanak.
E tekintetben a legfontosabb megoldandó feladat minden bizonnyal az,
hogy a rendszer képes legyen eltárolni a tartalomegyüttes bizonyos jóváhagyott állapotainak sorát.
Ez a régi rendszereken még fájlonként történt,
a modern megközelítésben ezzel szemben a teljes tartalomegyüttes
egy-egy pillanatképe, snapshotja jelenti az atomi egységet.
További alapvető feladat, amelyet egy ilyen rendszernek biztosítania kell, a párhuzamos fejlesztés támogatása.
A régi rendszerekben még egy-egy fájl központilag történő lockolása volt hivatott biztosítani azt,
hogy több fejlesztő ne írja fölül egymás munkáját.
A modern rendszerekben a párhuzamos fejlesztés külön ágakon, azaz brancheken párhuzamosan történik.

Jelenleg a legelterjedtebb verziókezelő a nyílt forráskódú \textit{Git}.
Filozófiájához hozzátartozik,
hogy a teljes verziókezelési folyamatot a helyi gépen kell elvégezni,
ezt lehet majd szinkronizálni más gépekre, például egy központi szerverre.
Mind a Git szervert, mind a lokális másolatot tárolónak (repository),
vagy röviden \textit{repó}nak nevezzük.
A Git elosztott rendszer,
használatához nincs feltétlenül szükség központi tárolóra.
Egy módosításcsomag publikálása két lépésből áll.
Elsőként a \textit{commit} művelettel létrehozzuk az új tárolt állapotot (snapshot);
a bevett terminológia szerint ezt is commitnak nevezik.
[TODO: a commit tartalma]
Alapértelmezetten az aktuális branch mentett állapota egyúttal az újonnan létrejött commitra ugrik.
Második lépésben a \textit{push} művelettel publikálhatjuk az adott branch (vagy branchek) állapotát egy szerverre.
A két lépés egymástól független,
és gyakran számos több commit létrehozása is megtörténik lokálisan,
mire sor kerülne a push műveletre.
A brancheken kívül \textit{tag}ek is létrehozhatók,
ezek hasonlóak a branchekhez,
a fő különbség, hogy véglegesek:
ha egyszer létrehoztunk egy adott commitra mutató taget,
az a továbbiakban immutábilisen erre a commitra fog mutatni
(ha mégis mozgatni szeretnénk, törölni kell, majd újra létrehozni).
Általában (bár ez megkerülhető) a fejlesztők a saját gépükön
egyszerre tárolják a teljes verziótörténetet,
beleértve a számukra érdekes brancheket, gyakran az összeset.
Amikor a \textit{clone} művelettel lemásolunk, leklónozunk egy repót,
alapértelmezetten az összes branch és tag letöltésre kerül a teljes történettel együtt.

\todoref{általános leírás vége}
%vagyis sok van, itt jön be a nehézség; kezdők...
%meta-verziókezelés, bár egy elágazó gráf, mindig van egy pillanatnyi meta-állapot

\section{A Git verziókezelő rendszer tárolási modellje}

\todoref{a git tárolási modellje}

\section{Tárolók átfogó állapotának kezelése}

\todoref{A git átfogó állapotairól}
%a subversion stb. még egy nagy (idődimenzióval kiegészített) fájlrendszerben tárolta a brancheket is,
%valójában nincs beépített branchkezelés, az ágak egyszerűen külön könyvtárakban vannak,
%az elágaztatási módszertanok működtetését a könyvtárak között alkalmazható
%merge-elő, összehasonlító stb. eszközök segítik (megjegyzés: ezt a git is tudja)

\section{A Git Chord megközelítése}

\todoref{a git-chord megközelítéséről}
% mentális modell
% a git szisztémájának természetes használata a háttérben

\section{Speciális esetek a Git Chord-ban}

\todoref{speciális esetek}
%submodule, git config módosításai stb.

\cleardoublepage

\chapter{Felhasználói dokumentáció}

\section{A parancssori program használata}

\subsection{Bevezetés}

A Git Chord egy POSIX Shellben írt kiegészítő a Git rendszerhez.
Egy repository átfogó metaállapotának teljes vagy részleges snapshotjait képes kezelni.
Ezek a snapshotok speciális (alapértelmezetten \texttt{chord/} prefixű) brancheken kerülnek tárolásra,
és kizárólag a Git rendszer natív képességeire építenek.

Egy-egy snapshot tárolja alapértelmezetten a branchek és tagek állapotát tárolja.
Ez a működés konfigurációval felülírható, finomhangolható,
illetve lehetőség van a staging area és a munkakönyvtár állapotának tárolására is.
Egyazon repository által kezelt több munkakönyvtár (\texttt{git-worktree}) tárolása jelenleg nem támogatott.
Lehetőség van a snapshotok létrehozására, böngészésére, törlésére, megosztására és letöltésére is.

\subsection{Telepítés és alapvető használat}

A Git Chord bármilyen POSIX-kompatibilis rendszeren futtatható.
Ezen felül egyetlen függősége maga a Git rendszer,
melynek legalább 2.0.0 verziója szükséges a teljeskörű használathoz.

A \texttt{bin} könyvtárban található \texttt{git-chord} szkriptfájl önmagában vagy a \texttt{git} parancs alparancsként is használható.
Az alparancsként való használathoz elengendő, ha a szkript változatlan névvel a végrehajtható fájlok keresési könyvtárainak egyikében található (POSIX rendszereken ezt a \texttt{PATH} környezeti változó reprezentálja).

Interaktív terminálban Bash héj esetén automatikus parancskiegészítést is használhatunk,
miután a \texttt{completion} könyvtárban található \texttt{git-chord-completion.bashrc} szkript be lett emelve (a `source` vagy `.` paranccsal).
Ajánlott ezt a beemelést a \texttt{bashrc} mechanizmusba helyezni,
hogy már a bejelentkezéskor lefusson.

Mindezek az inicializáló lépések az \texttt{install} könyvtárban található szriptek valamelyikével is elvégezhetők.
Az itt található \texttt{netinstall-user.sh} szkript a letöltéstől a teljes telepítésig elvégzi a folyamatot.
Hálózati elérés és a \texttt{curl} eszköz megléte esetén így a teljes telepítés elvégezhető egyetlen paranccsal:

\begin{minted}[breaklines]{shell}
curl -s https://raw.githubusercontent.com/davidsusu/git-chord/main/install/netinstall-user.sh | sh
\end{minted}

Sikeres telepítés után a programot önállóan (\texttt{git-chord})
vagy a \texttt{git} alparancsaként (\texttt{git chord}) is használhatjuk.
A plusz argumentumok kezelése és a működés a két esetben ugyanaz.
A továbbiakban mindig az alparancsos stílust fogom használni.

Használatba vétel előtt érdemes lehet ellenőrizni a telepített verziót:

\begin{minted}{shell}
git chord version
\end{minted}

Vagy röviden:

\begin{minted}{shell}
git chord -v
\end{minted}

Az első esetben a \verb|git-chord| program \texttt{version} alparancsát,
a második esetben a \verb|-v| POSIX kapcsolót használtuk.

További alapvető alparancs a \verb|help|, mely egy általános leírást ír ki,
magába foglalva az elérhető alparancsokat, kapcsolókat, paramétereket és egyéb információkat.
Ugyanerre a \verb|-h| kapcsoló is használható.

Az egyes alparancsokhoz saját leírás is elérhető,
ha az alparancs után további paraméterként megadjuk a \verb|help| szót
(vagy a \verb|-h| kapcsolót).
Például:

\begin{minted}{shell}
git chord snapshot help
\end{minted}

Részlet a kimenetből:

\begin{minted}{text}
git chord snapshot

Usage:
  git chord snapshot [<name>]

  git chord snapshot - <branch>

Example:
  git chord snapshot my-snapshots

Description:
  Creates and saves a snapshot of the entire state of [...]
\end{minted}

Ha a terminál támogatja az ANSI formázási escape szekvenciákat,
akkor a kimenet alapértelmezetten formázva jelenik meg.
Ez letiltható a \verb|--no-color| opció használatával.
(Általában a logikai paraméterek esetében a \verb|--no-| prefix tiltást,
a sima \verb|--| prefix pedig ellenkezőleg, kikényszerítést jelent.)
További kimenet-opció a \verb|--markdown|, mely a Markdown formátumú kimenetet kapcsolja be (illetve a \verb|--no-markdown| letiltja).
Egyes parancsok esetében lehetőség van bővebb kimenet bekapcsolására a \verb|--verbose|,
illetve tiltására a \verb|--no-verbose| opció használatával.

A normál kimenetek a standard kimenetre ($1$) kerülnek nyomtatásra,
míg a hibaüzenetek a hibakimenetre ($2$).

A \verb|--defaults| logikai opció bármely parancs esetében használható,
ekkor a programba beépített alapértelmezett konfigurációs értékek töltődnek be.
Számos parancsnál használható továbbá az \verb|--all| logikai opció,
pontos jelentése azonban specifikus
(a parancs által célzott objektumtípusból az összeset fogja használni).

A műveletet végrehajtó parancsok esetén lehetőség van a várható működés szimulálására
a művelet tényleges végrehajtása nélkül.
Ehhez a \verb|--dryrun| kapcsolót kell megadni.

A \texttt{git chord} parancsok háromféle státuszkóddal léphetnek ki:

\begin{quote}
\begin{description}
    \item[0:] Normál futtatáskor, siker esetén.
    \item[1:] A műveletet meghiúsító hiba esetén.
    \item[2:] Szimulált működés esetén (\verb|--dryrun|), ha nincs hiba.
\end{description}
\end{quote}

\subsection{Konfiguráció}

A Git Chord működése konfigurálható értékek megadásával szabható testre.
Minden ilyen értékhez tartozik egy konfigurációs kulcs.
A kulcs pontokkal választott szavakból áll, például: \verb|trackers.name| .

A konfiguráció több szinten kezelhető, melyek kulcsonként rendre felülírják egymást.
A szintek a következők:

\begin{description}
    \item[Alapértelmezések:]
        Ezek a programba beépített kezdőértékek.
        Ha valamely konfigurációs kulcs sehol máshol nincs definiálva, az itt meghatározott érték lesz használva.
        Példa: \verb|trackers.name main| .
    \item[Git config:]
        A Git rendszerben tárolt konfigurációban \verb|chord.| prefixszű kulcsokkal lehet konfigurációs értékeket definiálni a Git Chordhoz.
        A Git saját konfigurációs mechanizmusa maga is többszintű,
        az aktuálisan érvényes értékek lesznek használva.
        Példa: \verb|chord.trackers.name=master| .
    \item[Parancssori opciók:]
        A konfiguráció egy-egy parancsnál is megadható.
        Az opciókat \verb|--| prefix vezeti be, ezt követi a kulcs neve.
        Az ajánlott konvenció itt a mínuszjelek használata a pont karakter helyett,
        a normalizáció során ugyanis az előbbi az utóbbira cserélődik.
        Nem logikai értéket váró kulcs esetén az értéket a név után új argumentumként (vagy pedig egyenlőségjellel választva) kell megadni.
        A logikai értéket váró kulcsok esetében alapértelmezetten kényszerítés történik,
        a \verb|--| helyett a \verb|--no-| prefixet használva pedig tiltás.
        Ismétlődés esetén a későbbi értékadás felülírja a korábbit.
        Példa: \verb|--trackers-name mybranch| .
\end{description}

A parancssorban néhány speciális opciót is használhatunk, melyek egyszerre több konfigurációs kulcshoz rendelnek értéket:

\begin{description}
    \item[\texttt{{-}{-}defaults}:]
        Nem vár értéket (logikai típusú).
        Kényszeríti az alapértelmezett értékek használatát.
        A később következő parancssori opciók viszont felülírják az alapértelmezést.
    \item[\texttt{{-}{-}fullstore}:]
        Nem vár értéket (logikai típusú).
        Az összes támogatott objektum tárolását engedélyezi, azaz teljeskörű mentést állít be.
    \item[\texttt{{-}{-}fullstore}:]
        Nem vár értéket (logikai típusú).
        Az összes támogatott objektum betöltését engedélyezi, azaz teljeskörű visszaállítást állít be.
    \item[\texttt{{-}{-}profile}:]
        Értékként egy profil nevét várja.
        Betölti a profilban definiált konfigurációs beállításokat.
        Bővebben lásd: \ref{subsec:profiles}.
\end{description}

Támogatott továbbá a speciális \verb|--| argumentum, mely után opciókat már nem lehet megadni, csak egyszerű argumentumokat (például további alparancs, érték, stb.).

A \verb|git chord config --default| paranccsal kilistázhatjuk az alapértelmezett konfigurációt
(ez akkor is működik, ha nem egy repository könytára alatt vagyunk).
A \verb|--default| opció nélkül használt \verb|git chord config| parancsokkal kezelhetjük a Git Chordnak a Git saját mechanizmusával tárolt konfigurációit
(ez csak a repository könytára alatt működik).
Ekkor a következő lehetőségek elérhetők:

\begin{description}
    \item[\texttt{git chord config} (vagy \texttt{git chord config list}):]
        Listázza az aktuálisan érvényes konfiguráció kulcs-érték párjait.
    \item[\texttt{git chord config get <key>}:]
        Kiírja a megadott \verb|<key>| kulcshoz rendelt értéket.
    \item[\texttt{git chord config set <key> <value>}:]
        A megadott \verb|<key>| kulcshoz beállítja a megadott \verb|<value>| értéket.
    \item[\texttt{git chord config reset <key>}:]
        Törli a \verb|<key>| kulcshoz rendelt értéket (azaz megszünteti a felülírást).
\end{description}

Vegyük például a következő parancsot:

\begin{minted}{shell}
git chord config set trackers.name mybranch
\end{minted}

Ez a \verb|trackers.name| kulcshoz a \verb|mybranch| értéket fogja rendelni az aktuális repositoryban.
Ehhez a repositoryhoz tartozó Git által kezelt konfigurációban a \verb|chord.trackers.name| kulcsot fogja használni.
Vagyis a fenti parancs megfelel ennek:

\begin{minted}{shell}
git config set chord.trackers.name mybranch
\end{minted}

\subsection{Profilok használata} \label{subsec:profiles}

A profilok konfigurációs felülírások egy halmazát nevesítik.
Megadásuk úgy történik, hogy az adott kulcsot kiegészítjük a \verb|profile.<name>.|
(tehát a Git konfigurációjában a \verb|chord.profile.<name>.|) prefixszel,
ahol a \verb|<name>| helyére kerül a profil neve.
Példa:

\begin{minted}{shell}
git chord config set profile.lorem.trackers.prefix lorem/
git chord config set profile.lorem.trackers.name ipsum
git chord config set profile.lorem.lightweighttags.store.enabled true
\end{minted}

Ezzel létrejön a \verb|lorem| nevű profil, mely három felülírást tartalmaz.
A profil betöltéséhez a \verb|--profile| opciót használhatjuk,
melynek paramétere a profil neve, ez esetben tehát \verb|lorem|.
Például:

\begin{minted}{shell}
git chord snapshot --profile lorem
\end{minted}{shell}

A profil meghatározása után megadott egyéb opciók a sorrendnek megfelelően
felülírják a profilban megadott értékeket
(tehát minden esetben az adott kulcsot érintő legutolsó opció nyer).
Egy parancson belül akár több profilt is betölthetünk,
ezek szintén a felsorolás rendjében fogják felülírni egymás (és az egyéb értékadások) hatását.

\subsection{Snapshotok létrehozása, visszaállítása és törlése}

A repó metaállapotáról a \verb|git chord snapshot| alparanccsal lehet snapshotot készíteni.
Például:

\begin{minted}{shell}
git chord snapshot
\end{minted}

A fenti parancs létrehoz egy snapshotot a repository aktuális metaállapotáról.
Ehhez a beállított speciális branchre (alapértelmezetten \verb|chord/main|) egy új commit fog kerülni,
ebben szerepel majd \verb|state.yaml| néven a snapshot YAML formátumú leírása.
Egy ilyen leírás például így nézhet ki (egyszerű esetben):

\begin{minted}{yaml}
timestamp: '2025-06-01T14:14:54Z'
branches:
    'main': '233f62b1bfbb0cbc401e1f8e497985c8cd614b7c'
annotatedTags: {}
head:
    pointingTo: 'main'
    ref: 'main'
    commitId: '233f62b1bfbb0cbc401e1f8e497985c8cd614b7c'
\end{minted}

A leírásba az a minimális tartalom kerül, ami szükséges az állapot visszaállításához,
kiegészítve a snapshot létrehozásának dátumával.
A leírás szabványos YAML, tehát bármilyen ezzel kompatibilis
külső eszközzel is olvasható, feldolgozható.
Azonban a Git Chord a YAML formátumnak egy résznyelvét használja,
és a visszaállításkor is erre a résznyelvre számít.
Emiatt snapshotok kézzel vagy külső eszközzel történő létrehozása
a Git Chord belső működésének tanulmányozása nélkül nem ajánlott.

A hivatkozott commitok (valamint a hivatkozott tree objektumokhoz automatikusan létrehozott commitok)
bekerülnek a snapshot commit szülő-commitjai közé.
Ez biztosítja (a Git hivatkozási szisztémáján keresztül),
hogy pusztán ennek a commitnak a hordozásával annak minden függősége is átadódik.
A snapshot commit first-parent commitja mindig a speciális branchen már meglévő utolsó commit.

A speciális branch a \verb|trackers.prefix| és \verb|trackers.name| kulcsú konfigurációs értékek összefűzésével kerül meghatározásra.
A \verb|trackers.prefix| módosítása csak körültekintés mellett ajánlott,
ugyanis egyúttal ez az opció határozza meg,
hogy a Git Chord mely brancheket tekintse speciálisnak
(azaz hagyja ki mindenképpen a mentésből).
Ha a branch még nem létezik, akkor előbb létrehozásra kerül,
mégpedig egy üres tartalmú inicializáló orphan committal együtt,
ez lesz a snapshot first-parent commitja.

A \verb|--trackers-name| parancssori opció helyett a nevet egyszerűen plusz normál argumentumként is megadhatjuk:

\begin{minted}{shell}
git chord snapshot hello
\end{minted}

Ha pedig plusz argumentumként egyszerűen egy mínuszjelet írunk,
egy további paraméterként egy teljes branchnevet adhatjuk meg (csak kivételes esetekben ajánlott):

\begin{minted}{shell}
git chord snapshot - lorem/ipsum
\end{minted}

Hogy pontosan mi kerül mentésre, azt a \ref{tab:store-config-keys} táblázatban felsorolt
konfigurációs értékek befolyásolják közvetlenül.

\begin{table}[h]
\captionsetup{justification=justified, singlelinecheck=false}
\begin{center}
\begin{tabular}{|l|l|p{5cm}|} 
 \hline
 \multicolumn{1}{|c}{\textbf{Kulcs}} &
 \multicolumn{1}{|c}{\textbf{Alapérték}} &
 \multicolumn{1}{|c|}{\textbf{Leírás}} \\
 \hline\hline
 
 \texttt{branches.store.enabled} & \texttt{true} & Branchek mentésének engedélyezése\vspace{0.5em} \\
 \texttt{branches.store.regex} & \texttt{.*} & Minta a mentendő branchek szűkítéséhez\vspace{0.5em} \\
 \texttt{annotatedtags.store.enabled} & \texttt{true} & Annotált tagek mentésének engedélyezése\vspace{0.5em} \\
 \texttt{annotatedtags.store.regex} & \texttt{.*} & Minta a mentendő annotált tagek szűkítéséhez\vspace{0.5em} \\
 \texttt{lightweighttags.store.enabled} & \texttt{false} & Lightweight tagek mentésének engedélyezése\vspace{0.5em} \\
 \texttt{lightweighttags.store.regex} & \texttt{.*} & Minta a mentendő lightweight tagek szűkítéséhez\vspace{0.5em} \\
 \texttt{head.store.enabled} & \texttt{true} & HEAD mutató mentésének engedélyezése\vspace{0.5em} \\
 \texttt{stagingarea.store.enabled} & \texttt{false} & A staging area mentésének engedélyezése\vspace{0.5em} \\
 \texttt{workingtree.store.enabled} & \texttt{false} & A munkakönyvtár állapota mentésének engedélyezése\vspace{0.5em} \\
 \texttt{fullstore} & \texttt{false} & Minden objektum mentésének kényszerítése \\
 
 \hline

\end{tabular}
\end{center}
\caption{A tárolást érintő konfigurációs kulcsok}
\label{tab:store-config-keys}
\end{table}

Láthatjuk, hogy alapértelmezetten minden branch mentésre kerül,
és ez finomítható a szűrést megadó reguláris kifejezés módosításával.
Lényeges, hogy az annotált és lightweight tagek kezelése egymástól független.

A \verb|fullstore| rendesen csak ad hoc parancssori opcióként használatos,
és felülírja a táblázatban fölötte szereplő összes értéket a már ismertett módon.
A logikai értékeket mind \verb|true|-ra,
a \verb|.regex| végződésű kulcsok étékeit pedig \verb|.*|-ra állítja.

A következő parancs például pluszban menteni fogja az \verb|a| betűvel kezdődő lightweight tageket,
valamint a staging area állapotát:

\begin{minted}{shell}
git chord snapshot --lightweighttags-store-enabled true --lightweighttags-store-regex 'a.*' --stagingarea-store-enabled
\end{minted}

(Megjegyzés: az \verb|--all| opció nem helyettesíti a \verb|--fullstore| opciót.
Az \verb|--all| akkor használható, amikor egyértelmű a parancs által célzott objektumtípus,
abból összes kerül majd kiválasztásra
(például listázáskor az összes branch).)

Tehát ha hirtelen egy teljes mentést szeretnénk készíteni a konfigurácó állapotától függetlenül,
a következő paranccsal egyszerűen megtehetjük:

\begin{minted}{shell}
git chord snapshot --fullstore
\end{minted}

Vagy egy külön branchre mentve:

\begin{minted}{shell}
git chord snapshot --fullstore full
\end{minted}

A mentett snapshotok visszaállítása az \verb|apply| alparanccsal történik.
Kiadhatjuk például egyszerűen az alábbi parancsot:

\begin{minted}{shell}
git chord apply
\end{minted}

Ekkor szintén a \verb|trackers.prefix| és \verb|trackers.name| kulcsú konfigurációs értékek összefűzésével
kerül meghatározásra a speciális branch,
és ennek a branchnek az utolsó snapshot commitja kerül visszaállításra.
Ha nincs ilyen, akkor a parancs hibával leáll.

A \verb|snapshot| alparancshoz hasonlóan itt is használhatjuk plusz normál argumentot a \verb|--trackers-name| helyett.
Például:

\begin{minted}{shell}
git chord apply hello
\end{minted}

Ez alapértelmezetten a \verb|chord/hello| branchen található utolsó snapshotot tölti be.
Ha nem a legutolsó snapshotot szeretnénk visszaállítani, akár a név megadása után, akár anélkül,
használhatjuk a sorszám-kapcsolót, azaz egy plusz argumentumot, mely egy mínuszjellel prefixelt nem-negatív egész szám
(a \verb|-0| jelenti a legutolsót, az \verb|-1| az utolsó előttit, és így tovább).
Például:

\begin{minted}{shell}
git chord apply hello -3
\end{minted}

Itt is használhatjuk a mínuszjel argumentumot, utána egy tetszőleges karakterlánc megadható,
melyet a Git rendszer commitként tud azonosítani.
Az előbbi parancsnak például megfelel a következő (feltéve, hogy a \verb|trackers.prefix| alapértelmezésre van állítva):

\begin{minted}{shell}
git chord apply - chord/hello~3
\end{minted}

A \ref{tab:apply-config-keys} táblázat részletezi a visszaállítás tartalmát érintő konfigurációs kulcsokat.

\begin{table}[h]
\captionsetup{justification=justified, singlelinecheck=false}
\begin{center}
\begin{tabular}{|l|l|p{4.5cm}|} 
 \hline
 \multicolumn{1}{|c}{\textbf{Kulcs}} &
 \multicolumn{1}{|c}{\textbf{Alapérték}} &
 \multicolumn{1}{|c|}{\textbf{Leírás}} \\
 \hline\hline
 
 \texttt{branches.apply.enabled} & \texttt{true} & Branchek visszaállításának engedélyezése\vspace{0.5em} \\
 \texttt{branches.apply.regex} & \texttt{.*} & Minta a visszaállítandó branchek szűkítéséhez\vspace{0.5em} \\
 \texttt{branches.apply.allowremove} & \texttt{true} & Branchek törlésének engedélyezése\vspace{0.5em} \\
 \texttt{branches.apply.allowadd} & \texttt{true} & Branchek létrehozásának engedélyezése\vspace{0.5em} \\
 \texttt{annotatedtags.apply.enabled} & \texttt{true} & Annotált tagek visszaállításának engedélyezése\vspace{0.5em} \\
 \texttt{annotatedtags.apply.regex} & \texttt{.*} & Minta a visszaállítandó annotált tagek szűkítéséhez\vspace{0.5em} \\
 \texttt{annotatedtags.apply.allowremove} & \texttt{true} & Annotált tagek törlésének engedélyezése\vspace{0.5em} \\
 \texttt{lightweighttags.apply.enabled} & \texttt{false} & Lightweight tagek visszaállításának engedélyezése\vspace{0.5em} \\
 \texttt{lightweighttags.apply.regex} & \texttt{.*} & Minta a visszaállítandó lightweight tagek szűkítéséhez\vspace{0.5em} \\
 \texttt{lightweighttags.apply.allowremove} & \texttt{false} & Lightweight tagek törlésének engedélyezése\vspace{0.5em} \\
 \texttt{head.apply.enabled} & \texttt{true} & HEAD mutató mozgatásának engedélyezése\vspace{0.5em} \\
 \texttt{head.apply.sticktocommit} & \texttt{false} & A HEAD mutató mindenképpen a commithoz ragaszkodjon\vspace{0.5em} \\
 \texttt{stagingarea.followhead} & \texttt{true} & A staging area alapból a HEAD-et kövesse\vspace{0.5em} \\
 \texttt{stagingarea.apply.enabled} & \texttt{false} & A staging area állapotának visszaállításának engedélyezése\vspace{0.5em} \\
 \texttt{workingtree.followstagingarea} & \texttt{true} & A munkakönyvtár állapota alapból a staging area-t kövesse\vspace{0.5em} \\
 \texttt{workingtree.apply.enabled} & \texttt{false} & A munkakönyvtár állapotának visszaállításának engedélyezése\vspace{0.5em} \\
 \texttt{fullapply} & \texttt{false} & Minden objektum visszaállításának kényszerítése \\

 \hline

\end{tabular}
\end{center}
\caption{A tárolást érintő konfigurációs kulcsok}
\label{tab:apply-config-keys}
\end{table}

Az \verb|apply| tokent tartalmazó kulcsokhoz tartozó alapértelmezett értékek
rendre megegyeznek a fentebb a \verb|store| tokent tartalmazó megfelelő kulcsok alapértelmezéseivel.
Ez azonban nem szükségszerű.
Ha a snapshotban szerepelnek olyan objektumok, amelyek visszaállítása nincs engedélyezve,
azok egyszerűen figyelmen kívül lesznek hagyva.
Ha azonban a snapshotban bizonyos objektumok nem szerepelnek,
de a repository aktuális állapotában igen,
és a visszaállítás vonatkozik rájuk,
akkor a megfelelő \verb|.allowremove| végződésű beállítás bekapcsolása esetén ezek törölve lesznek.

Ha a snapshotból olyan objektum érkezik, melynek neve egyezik a repository állapotában található más típusú objektummal,
akkor a Git Chord először megpróbálja törölni az aktuális objektumok,
feltéve, hogy ez engedélyezve van.
A Git rendszerben a tagek immutábilis objektumok.
Ezért a Git Chord ezek módosítását egy törlés és egy létrehozás szekvenciájaként értelmezi,
vagyis, a branchekkel ellentétben,
a tagek módosítááshoz a törlés engedélezése is szükséges.

A \verb|head.apply.sticktocommit| beállítás akkor válik érdekessé,
ha a HEAD mutató szerepel a snapshotban, és egy branchre mutat,
azonban ez a branch nincs mentve a snapshotva.
Ilyenkor természetesen könnyen előfordulhat, hogy a branch elmozdult ahhoz az állapothoz képest,
ahol a snapshot létrehozásakor volt.
A YAML-fájlban ilyenkor a HEAD-hez a branch és a hozzá tartozó commit azonosítója is szerepel
(persze detached head is menthető, ekkor a branchnév hiányozni fog).
A visszaállítás folyamatában ez esetben a következő dilemma lép föl:
a HEAD mutatót a branchre kell visszaállítani, függetlenül attól, hogy az már elmoudult,
vagy pedig mindenképp az eredeti commitra kell mutasson?
Az alapértelmezett működés, hogy a HEAD mutató a branchre lesz állítva.
A \verb|head.apply.sticktocommit| engedélyezésével azonban más szabály lép érvénybe,
és a HEAD mutató detached head állapotba kerülve közvetlenül az eredeti commitra fog mutatni.
Ez az opció jól jönn akkor, ha semmiképp nem akarjuk,
hogy a HEAD mutató elvándoroljon ahhoz képest, ahogy lementettük.

A \verb|stagingarea.followhead| engedélyezése biztosítja,
hogy a staging area alapértelmezetten kövesse a HEAD-et, akkor is, ha nem szerepel a snapshotban.
Ez természetes módon igazodik a Git rendszer saját működéséhez,
ugyanakkor olyasmit is változtat a repository állapotán, ami nem volt mentve a snapshotba.
Ha a staging are jelenleg nem mozgatható (például nem tiszta),
akkor a művelet hibával végződik.

A \verb|workingtree.followstagingarea| engedélyezése, az előzőhöz hasonlóan,
azt biztosítja, hogy a munakönyvtár állapota kövesse a staging area állapotát.
Ha a munkakönyvtárban indexeletlen módosítások vannak,
a művelet hibával végződik.

A snapshotokat utólag törölhetjük is, erre szolgál a \verb|delete| alparancs.
A parancs szitaxisa teljesen megegyezik az \verb|apply| alparancsnál látottakkal,
ugyanúgy kiadhatjuk például további argumentumok nélkül is:

\begin{minted}{shell}
git chord delete
\end{minted}

Ekkor az előre beállított speciális branch utolsó snapshot commitja fog törlődni.
Illetve ugyanúgy használhatjuk a plusz paramétereket is, például:

\begin{minted}{shell}
git chord delete hello -4
\end{minted}

Vagy:

\begin{minted}{shell}
git chord delete - chord/hello~4
\end{minted}

Ha a mínusz argumentumot használjuk, tetszőleges commitot megcímezhetünk.
Ekkor azonba a Git Chord előbb ellenőrzi,
hogy a megadott commit valóban egy metasnapshot-e,
ellenkező esetben a törlés hibával meghiúsul.

A törléskor a snapshot commit kifűzésre kerül a commitok láncolatából.
Ha ez volt az egyetlen commit, akkor a speciális branch törlődik.
Ha ez volt a legutóbbi commit, akkor a branch mutatója átállításra kerül
a törlendő commit first-parent commitjára.
Egyéb esetben a későbbi commitok átírására van szükség,
ami úgy történik, hogy a másolatokban a first-parent azonosító kerül lecserélésre,
először a törölt commitéra, majd mindig az előzőleg létrehozott másolatéra;
végül a speciális branch mutatója az utolsó másolt commitra kerül áthelyezésre.
Hasonló eredmény egyszerűbb esetekben egy megfelelően paraméterezett \verb|git rebase|
paranccsal is elérhető,
de ez nem ajánlott, különös tekintettel a sok szülő-commitra.

A Git rendszer szemszögéből a Git Chord speciális branchei is csak szokványos branchek.
A törölt állapotok és a speciális branchek régebbi állapotának visszahozása
még a törlés után is lehetséges a \verb|git reflog| használatával,
ha nem történt még olyan tisztítás a repository objektumai között,
ami ezek tartalmait végleg törölte volna.

\subsection{Snapshotok böngészése}

A mentett snapshotok tartalma megtekinthető a \verb|show| alparanccsal.
Ennek szintaxisa teljesen egyezik a \verb|apply| és \verb|delete| alaprancsoknál látottal.
Ha további argumentumok nélkül adjuk ki, akkor az előre beállított speciális branch
utolsó snapshot commitját fogja megmutatni:

\begin{minted}{shell}
git chord show
\end{minted}

Vagy további argumentumokkal valamely másik snapshotét:

\begin{minted}{shell}
git chord show hello -2
\end{minted}

Ha nem hasznljuk a \verb|--verbose| opciót, akkor egyszerűen a YAML-fájl tartalma íródik ki.

Az eddig mentett összes snapshot listáját megkaphatjuk a \verb|list| alparanccsal:

\begin{minted}{shell}
git chord list
\end{minted}

Ennek lehetséges kimenete (egyetlen mentett snapshot esetén):

\begin{minted}{text}
0ebb7da (chord/main) Chord repository state at 2025-06-01T14:14:54Z
\end{minted}

A snapshotok időbeli sorrendben listázódnak, a legújabbtól a legrégebbiig.

Használhatjuk (egymástól függetlenül) a névmegadás és a sorszám lehetőségét is:

\begin{minted}{shell}
git chord list hello -4
\end{minted}

Ha sorszámot is megadunk, akkor csak az annak megfelelő snapshot és az annál régebbiek listázódnak.

Ha megadjuk az \verb|--all| opciót, akkor egymás után minden speciális branch külön listázásra kerül.

A \verb|state| alparanccsal a repository aktuális állapotát írhatjuk ki:

\begin{minted}{shell}
git chord state
\end{minted}

Azt a YAML tartalmat fogja kiírni, ami mentésre kerülne, ha most adnánk ki a \verb|snapshot| alparancsot.

A \verb|diff| alparanccsal két mentett snapshotot hasonlíthatunk össze,
vagy pedig az aktuális állapotot egy snapshottal.
Előbbihez két snapshotot, utóbbihoz értelemszerűen csak egyet kell megadni.
Ha nem adunk meg egyetlen az összehasonlítandó snapshotot sem,
akkor az aktuális állapotot lesz összehasonlítva a legutóbbi snapshottal.

Az alparancs szintaxisa hasonló, mint amit az \verb|apply|, \verb|delete| és \verb|show|
esetében láthattunk,
azzal a különbséggel, hogy egy második snapshot is megadható.
A \ref{tab:diff-commands} táblázat mutat néhány példát a paraméterek használatára.

\begin{table}[h]
\captionsetup{justification=justified, singlelinecheck=false}
\begin{center}
\begin{tabular}{|l|c|c|} 
 \hline
 \multicolumn{1}{|c}{\textbf{Diff parancs}} &
 \multicolumn{1}{|c}{\textbf{A}} &
 \multicolumn{1}{|c|}{\textbf{B}} \\
 \hline\hline
 
 \texttt{git chord diff} & \texttt{chord/main} & \textit{aktuális} \\
 \texttt{git chord diff -2} & \texttt{chord/main\~{}2} & \textit{aktuális} \\
 \texttt{git chord diff hello -2} & \texttt{chord/hello~{}2} & \textit{aktuális} \\
 \texttt{git chord diff hello -2 -3} & \texttt{chord/hello\~{}2} & \texttt{chord/hello\~{}3} \\
 \texttt{git chord diff lorem ipsum} & \texttt{chord/lorem} & \texttt{chord/ipsum} \\
 \texttt{git chord diff lorem -2 ipsum} & \texttt{chord/lorem\~{}2} & \texttt{chord/ipsum} \\
 \texttt{git chord diff lorem -ipsum -3} & \texttt{chord/lorem} & \texttt{chord/ipsum\~{}3} \\
 \texttt{git chord diff lorem -2 ipsum -3} & \texttt{chord/lorem\~{}2} & \texttt{chord/ipsum\~{}3} \\
 \texttt{git chord diff - chord/hello\~{}3} & \texttt{chord/hello\~{}3} & \textit{aktuális} \\
 \texttt{git chord diff - chord/lorem chord/ipsum} & \texttt{chord/lorem} & \texttt{chord/ipsum} \\
 
 \hline
 
\end{tabular}
\end{center}
\caption{Snapshotokat összehasonlító parancsok jelentése}
\label{tab:diff-commands}
\end{table}

Alapból a Git alapértelmezett \verb|diff| algoritmusának kimenete lesz megjelenítve a két YAML tartalomra:

\begin{minted}{text}
--- /tmp/git-chord.255120140/state1.yaml	2025-06-01 22:42:42.252497000 +0200
+++ /tmp/git-chord.255120140/state2.yaml	2025-06-01 22:42:42.252497000 +0200
@@ -1,8 +1,8 @@
-timestamp: '2025-06-01T14:14:54Z'
+timestamp: '2025-06-01T20:42:42Z'
 branches:
-    'main': '233f62b1bfbb0cbc401e1f8e497985c8cd614b7c'
+    'main': 'bb76ba05fd5b141cdfe2706523ccae483d4a4cad'
 annotatedTags: {}
 head:
     pointingTo: 'main'
     ref: 'main'
-    commitId: '233f62b1bfbb0cbc401e1f8e497985c8cd614b7c'
+    commitId: 'bb76ba05fd5b141cdfe2706523ccae483d4a4cad'
\end{minted}

A \verb|--verbose| opció használatával egy szövegesen megmagyarázott kimenetet kapunk, például:

\begin{minted}{text}
Diff between 'chord/main' and the current state

Branches
   * Branch changed: main, from 233f62b1bfbb0cbc401e1f8e497985c8cd614b7c to bb76ba05fd5b141cdfe2706523ccae483d4a4cad

HEAD
   * HEAD changed, from main (233f62b1bfbb0cbc401e1f8e497985c8cd614b7c) to main (bb76ba05fd5b141cdfe2706523ccae483d4a4cad)
\end{minted}

\subsection{Snapshotok megosztása}

A Git Chord által mentett snapshotok a Git rendszer saját \verb|push| és \verb|fetch|
mechanizmusain keresztül is viszonylag egyszerűen megoszthatók.

A teljesség kedvéért a Git Chord maga is támogatást nyújt ezekhez a műveletekhez.
Kezdő felhasználók számára mindenképpen inkább ezek a használata ajánlott.

A remote repository a \verb|trackers.remotes.default| opción keresztül állítható be.
A \verb|trackers.remotes.allowautoassociate| opció engedélyezi a remote beállítások automatikus
létrehozását a helyi repositoryban (alapértelmezetten engedélyezve van).
Ekkor szükség esetén létrejön a megfelelő remote branch a beállított remote repositoryhoz rendelve.

A \verb|push| alparanccsal kiküldhetjük a snapshotokat a remote repositoryba:

\begin{minted}{shell}
git chord push
\end{minted}

Ekkor az alapértelmezett speciális branch kerül pusholásra.
A Git Chord által kezelt összes speciális branch pusholásához
az \verb|--all| opciót kell megadni:

\begin{minted}{shell}
git chord push --all
\end{minted}

A Git Chord \verb|pull| alparancsa a háttérben a Git rendszer \verb|pull| parancsát hívja meg,
és annak alapértelmezéseit használja.
Tehát először egy \verb|fetch| műveletet hajt végre a remote branchre,
majd alapértelmezésben egy \verb|rebase| művelettel applikálja azt a Git Chord által kezelt branchre.

Pullozáskor csak a snapshotok mentődnek le a helyi repositoryba,
semmilyen visszaállítás nem történik.

A \verb|pull| alparancs használata hasonló a \verb|push|-éhoz.
Kiadhatjuk további paraméterek nélköl:

\begin{minted}{shell}
git chord pull
\end{minted}

Ekkor a beállított speciális branchre kerülnek letöltésre a remote repository megfelelő snapshotjai.
Ugyanúgy használhatjuk az \verb|--all| kapcsolót is:

\begin{minted}{shell}
git chord pull --all
\end{minted}

Ekkor a letöltés a Git Chord által kezelt minden branchre külön-külön szekvenciálisan megtörténik.

\subsection{Egyéb parancsok}

A parancssori felület egyes funkciói gépi feldolgozásra lettek létrehozva.
Ezek a funkciók a \verb|spec| alparancson, annak további alparancsain keresztül érhetők el.
A \verb|commands| alparancs mutatja az elérhető parancsokat:

\begin{minted}{shell}
git chord spec commands
\end{minted}

Ennek kimenete jelenleg:

\begin{minted}{text}
help
version
config list get set reset
snapshot
apply
state
show
list
diff
delete
push
pull
spec commands options
\end{minted}

A parancslista kétszintű
Az elő oszlop a legfelsőbb szintű alparancsokat tartalmazza.
Egy-egy sorban a további szavak ennek a további alparancsai.

Az \verb|options| alparancs pedig az összes elérhető parancssori opciót listázza:

\begin{minted}{shell}
git chord spec options
\end{minted}

Interaktív terminálban ezeket a parancsokat használja az automatikus parancskiegészítés szkriptje is.

\section{A webes felület használata}

TODO

\cleardoublepage

\chapter{Fejlesztői dokumentáció}

\section{A parancssori \texttt{git-chord} program}

\subsection{A projekt forrásrepójának szerkezete}

A forrásrepó főkönyvtárában az alábbiakat találjuk:

\begin{description}
    \item[\faFolder~bin] --
        A fő futtatható állományok könyvtára.
        Jelenleg egyetlen fájl található itt, a monolitikus \verb|git-chord| POSIX shell szkript.
    \item[\faFolder~completion] --
        Az automatikus parancskiegészítéshez kapcsolódó fájlok könyvtára.
        Jelenleg egyetlen fájl található itt, a \verb|git-chord-completion.bashrc| Bash szkript.
    \item[\faFolder~install] --
        A telepítést segítő futtatható fájlokat tartalmazó könyvtár.
    \item[\faFolder~maintain] --
        Különféle fejlesztői segédszkripteket tartalmazó könyvtár.
    \item[\faFolder~packaging] --
        A csomagkezeléssel, terjesztéssel kapcsolatos fájlok találhatók itt.
        Jelenleg csak a \verb|.deb| csomag buildelése támogatott.
    \item[\faFolder~test] --
        A teszteseteket, a tesztvezérlő szkripteket és a tesztfuttatási könyezeteket
        tartalmazó könyvtár.
    \item[\faFile~LICENSE] --
        A licenszfájl.
        A repó teljes tartalmára a MIT licensz vonatkozik.
    \item[\faFile~README.md] --
        A README fájl, ami a projekt általános leírását tartalmazza.
        Ez egy generált fájl, nem szabad kézzel módosítani.
    \item[\faFile~regen.sh] --
        POSIX shell szkript, mely újrabuildeli a repó generált tartalmait.
\end{description}

A \verb|regen.sh| szkript valójában a \verb|maintain/regenerate-sources.sh| szkriptet hívja meg,
ami jelenleg egyetlen generálást futtat, mégpedig a \verb|maintain/gen-README.md.sh| fájlt.
Utóbbi gyártja le a főkönyvtárban lévő \verb|README.md| fájl tartalmát.
Ehhez a mellette található \verb|gen-doc-part.md.sh| fájlt hívogatja,
mely egy-egy résztartalom (\verb|general|, \verb|help|, \verb|subcommands| stb.) legyártásáért felel.
Az ugyyanitt lévő \verb|gen-manpage.sh| szkript szintén a \verb|gen-doc-part.md.sh| fájlt hívogatja,
az egyes alparancsok saját help kimeneteinek lekéréséhez,
és egy GROFF formátumú kiemenetet ad.
Szintén itt a \verb|check-script.sh| forráskód-ellenőrzést futtat a \verb|git-chord| szkriptre
a ShellCheck segítségével, mely egy statikus ellenőrző program POSIX shell szkriptekhez.

Az \verb|install| könyvtárban néhány telepítőszkript található.
A \verb|netinstall-user.sh| fő használati esete, ha hálózatról töltjük le;
a repó többi részét már maga klónozza le,
és rögtön futtatja az ugyanebben a mappában található
\verb|install-user-bashrc-path.sh| és
\verb|install-user-manpage.sh| szkripteket.
Az előbbi Bash shell esetén a bashrc konvenció használatával végzi el a telepítést,
a megfelelő helyre elhelyezve a hivatkozást a fő futtatható állományt és az automatikus parancskiegészítést.
Az utóbbi a manuáloldalt telepíti.
Ugyanitt található még a \verb|install-global-usrlocalbin.sh| szkript,
ez az előbbiektől eltérően (ahol felhasználóhoz kötve történt a telepítés)
rendszerszinten, globálisan telepíti a futtatható állományt,
bemásolva azt a \verb|/usr/local/bin| könyvtárva;
futtatásához értelemszerűen rendszergazdai jog szükséges.

A \verb|packaging/build-all.sh| POXIS shell szkript tetszőleges számú csomagépítési mechanizmusra van felkészítve.
A legyártott csomagok a \verb|packaging/out| könyvtárba kerülnek.
Maguk a mechanizmusok a \verb|packaging/builders| alkönyvtáraiban foglalnak helyet,
és egy \verb|build.sh| nevű futtatható állományt kell tartalmazniuk.
Egyelőre csak a \verb|deb| builder lett implementálva.
Ez létrehoz egy Debian telepítőcsomagot, melybe elhelyezi a \verb|git-chord| szkriptet,
illetve a manuáloldalt, melyet helyben generál le a \verb|README.md| fájl alapján.

A tesztesetekkel később részletesebben foglalkozom.

\subsection{A fő shell szkript felépítése}

A \verb|bin/git-chord| egy POSIX-kompatibilis shell szkript.

Felépítésének váza a következő:

\begin{itemize}
    \item header
    \item definíciók
    \item segédfüggvények
    \item beállítások, opciók és paraméterek feldolgozása, alparancs megállapítása
    \item formázók beállítása a kimenet-beállítások alapján
    \item az opcióktól is függő függvények
    \item (teljesen vagy feltételesen) repófüggetlen alparancsok kiszolgálása (help, version, default-kezelés stb.)
    \item git repó ellenőrzése
    \item repófüggő alparancsok kiszolgálása
\end{itemize}

A header rész a következő két sorból áll:

\begin{minted}{shell}
#!/bin/sh
# shellcheck disable=SC2016,SC2086,SC2154
\end{minted}

Az első sor a shebang komment, ami meghatározza, hogy a fájl közvetlen futtatásához mely interpreter használandó.
Maga a POSIX szabvány érdekes módon nem nyilatkozik az interpreter fájlrendszerbéli helyéről,
sőt, arról nyilatkozik, hogy a szkriptet telepítéskor kell meghatározni,
hogy az adott rendszer interpretere hol található
(vagyis valójában a POSIX shell szkriptek nem teljesen dhordozhatók).
A gyakorlatban komoly viták vannak akörül, végülis mit érdemes megadni,
a fentin kívül egy másik gyakori jelölt \verb|#!/usr/bin/env sh|.
Mindkét verzió mellett szólnak pro és kontra érvek,
én minden POSIX shell szkriptben a fenti shebang sort használom.

A második sor ShellChecker eszközt számára némít el egyes hibaüzeneteket
a hozzájuk tartozó szabályok felsorolásával:

\begin{description}
    \item[\texttt{SC2016}] --
        Ez a szabály hivatott kiszűrni azt a hibát, amikor egy dinamikusan megadott sztring köré
        tévedésből egyszeres idézőjeleket teszünk
        (például \verb|'Név: ${name}'|).
        Ezzel azonban hamis riasztást generál például az ilyen helyeken: \verb|'`%s`'|.
    \item[\texttt{SC2086}] --
        Ez a szabály kiszűri azokat az eseteket, amikor egy parancs argumentumainak felsorolásakor
        lehagyjuk az idézőjeleket egy változóval megadott argumentum széleiről.
        Ez hamis riasztást generál, ha az elhagyás szándékos,
        vagyis valóban azt akarjuk, hogy a változó tartalma több argumentra eshessen szét.
    \item[\texttt{SC2154}] --
        Ez a szabály értesít róla, ha definiálatlan változót akarunk használni (ha az nem csupa nagybetűs).
        Ez hamis riasztást generál, ha például a változót dinamikusan hoztuk létre.
\end{description}

A definíciós részben definiált konstansok meghatározzák a program verzióját,
az elérhető parancsokat és alparacsokat leírásukkal együtt,
az elérhető konfiguráciüs kulcsokat alapértelmezett értékükkel és leírásukkal,
illetve egyéb segédadatokat.

Ezután a tiszta, általános jellegű segédfüggvények definíciói következnek,
például karakterláncok kezeléséhez,
parancssori argumentumok feldolgozásához,
a YAML formátum kezeléséhez
és így tovább.

Például egy parancssori opció konfigurációs kulccsá történő normalizálásához:

\begin{listing}[H]
\begin{minted}{shell}
normalizeKey() { # $1: rawKey
    printf '%s' "$1" | sed -E 's/^\-\-(no\-)?//i' | tr '-' '.' | tr '[:upper:]' '[:lower:]'
}
\end{minted}
\caption{Parancssori opció normalizálása konfigurációs kulccsá}
\end{listing}

Vagy átmeneti könyvtárak létrehozásához:

\begin{listing}[H]
\begin{minted}{shell}
createTmpDir() (
    sysTmpDir="$TMPDIR"
    if [ -z "$sysTmpDir" ]; then
        sysTmpDir="/tmp"
    fi
    pathPrefix="${sysTmpDir}/git-chord.$( date '+%N' )"
    path="$pathPrefix"
    i='1'
    while [ -e "$path" ]; do
        path="${pathPrefix}."
        i=$(( i + 1 ))
    done
    mkdir -p "$path"
    printf '%s\n' "$path"
)
\end{minted}
\caption{Átmeneti könyvtár készítése, POSIX-kompatibilis}
\end{listing}

Ezután történik a Git konfigurációjának feldolgozása, és összefésülése az alapértelmezett értékekkel.
Ha szerepel olyan \verb|chord.| prefixű kulcs, amely nincs definiálva,
a szkript figyelmeztetést ír ki, de nem áll le.

Ezután következik a parancssori argumentumok feldolgozása.
Itt három módot különböztetünk meg, az \textit{alap módot}, az \textit{érték módot} és a \textit{nyers módot}.

Alap módban fogadhatunk egy \verb|--| prefixű konfigurációs opciót, normál paramétert (például alparancs, érték), vagy pedig a speciális \verb|--| argumentumot.
Ha opció érkezett, megjegyezzük a konfigurációs kulcsot, és átkapcsolunk érték módba.
Ha normál paraméter érkezett, behelyezzük a normál paraméterek tömbjébe.
Ha a \verb|--| argumentum érkezett, átkapcsolunk nyers módba.

Érték módban beolvassuk a következő argumentumot, és értékül adjuk a megjegyzett konfigurációs kulcshoz.
Ha nincs több paraméter, végzetes hibát dobunk.
Ellenkező esetben visszakapcsolunk alap módba, és folytatjuk a feldolgozást.

Nyers módban beolvassuk a következő argumentumot, ha van,
és behelyezzük a normál paraméterek tömbjébe.
Továbbra is nyers módban maradunk
(a nyers mód már nem kapcsolható vissza).

Ha a normál paraméterek tömbje ezután nem üres, akkor az első eleme lesz az alparancs,
ezt a \verb|$command| változóban eltároljuk.
Különben az alapértelmezett \verb|snapshot| alparancs lesz használva.
Ha a tömb legalább két értéket tartalmaz, akkor a másodikat további alparancsként
megjegyezzük a \verb|$subcommand| változóban.

A könnyebb használat és a jobb teljesítmény érdekében a konviguráció végső effektív értékeit
dinamikusan létrehozott változókba is kihelyezzük.
Eredetileg az ezt elvégző kódrészlet így nézett ki:

\begin{listing}[H]
\begin{minted}{shell}
while IFS= read -r configLine; do
    configValue="$( printf '%s\n' "$configLine" | cut -d ' ' -f 2- )"
    outVariableName="configValue_$( printf '%s\n' "$configLine" | cut -d ' ' -f 1 | tr '.' '_' )"
    eval "${outVariableName}='$( escapeValue "$configValue")'"
done <<EOF
$( printf '%s\n' "$config" )
EOF
\end{minted}
\caption{Eredeti eval loop}
\end{listing}

A POSIX shell (a Bash nyelvvel ellentétben) nem ad beépített lehetőséget változók dinamikus értékadására,
így az \verb|eval| használata elkerülhetetlen.
Javíthatunk a teljesítményen, ha az összes értékadást egyetlen \verb|eval| utasításba tömörítjük.
Továbbá, mivel nem módosítunk a külső környezeten,
a ciklus bemenetét adó parancsot is kihelyezhetjük egy \verb@|@ jel elé.
A javított verzió így néz ki:

\begin{listing}[H]
\begin{minted}{shell}
eval "$(
    printf '%s\n' "$config" | sed -E ':loop; s/^(\w+)\./\1_/; t loop' | while IFS=' ' read -r outVariableName configValue; do
        printf "configValue_%s='" "$outVariableName"
        escapeValue "$configValue"
        printf "'\n"
    done
)"
\end{minted}
\caption{Optimalizált eval loop}
\end{listing}

A későbbiekben tehát változókon keresztül is tudunk hivatkozni az effektív konfigurációs értékekre,
például a \verb|trackers.name| értékére így: \verb|$configValue_trackers_name| .

Ezután beállítjuk a formázó karakterláncokat,
figyelembe véve, hogy a \verb|color| opció be van-e kapcsolva.
Utóbbi esetben ANSI formázási szekvenciákat fogunk használni,
különben a formázó sztringek üresek lesznek.

Ezt követik azok a függvények, melyek függnek a konfigurációs értékektől.
Például:

\begin{listing}[H]
\begin{minted}{shell}
getLength() { # $1: string
    if checkTruthy "$configValue_color"; then
        printf '%s' "$1" | sed -E 's/\x1b\[[0-9;]*m//g' | wc -m
    else
        printf '%s' "$1" | wc -m
    fi
}
\end{minted}
\caption{Szöveg hosszának lekérése az ANSI formázások feltételes figyelembevételével}
\end{listing}

Ezt követően ellenőrizzük, hogy a \verb|$subcommand| változóban
a \verb|help| (vagy \verb|-h|) érték szerepel-e.
Ez esetben kiírjuk a \verb|$command| változó által meghatározott alparancs help kimenetét és kilépünk.

Különben a futás folytatódik az első alparancs feldolgozásával,
megengedve, hogy esetleg nem vagyunk Git repositoryban.
Ilyenkor futnak le azok az esetek, amikor olyan alparancsot futtatunk,
amely eleve nem igényli a repository meglétét (például \verb|help|),
vagy például ha a \verb|--defaults| opció be lett állítva.

Egyéb esetben szükség lesz a repository meglétére,
a következő lépésben ezt ellenőrizzük a következő módon:

\begin{listing}[H]
\begin{minted}{shell}
if ! git rev-parse --git-dir > /dev/null 2>&1; then
    printf '%sERROR: %s%s\n' "$formatError" 'Not a git repository' "$formatErrorEnd" >&2
    exit 1
fi
\end{minted}
\caption{Git repository meglétének ellenőrzése}
\end{listing}

Végül az utolsó szakaszban dolgozzuk fel az alparancsot, már a repository meglétét feltételezve.
Ezt tekinthetjük a főrésznek a szkriptben.

[TODO]


először:

\begin{listing}[H]
\begin{minted}{shell}
tempYamlFile1="$( mktemp )" # FIXME: mktemp is not POSIX
printf '%s\n' "$stateYaml1" > "$tempYamlFile1"
tempYamlFile2="$( mktemp )" # FIXME: mktemp is not POSIX
printf '%s\n' "$stateYaml2" > "$tempYamlFile2"
if checkTruthy "$configValue_color"; then
    diff -u -U3 --color "$tempYamlFile1" "$tempYamlFile2"
else
    diff -u -U3 "$tempYamlFile1" "$tempYamlFile2"
fi
rm "$tempYamlFile1"
rm "$tempYamlFile2"
\end{minted}
\caption{Átmeneti fájlok kezelése előtte}
\end{listing}

majd:

\begin{listing}[H]
\begin{minted}{shell}
tempDir="$( createTmpDir )"
tempYamlFile1="${tempDir}/state1.yaml"
printf '%s\n' "$stateYaml1" > "$tempYamlFile1"
tempYamlFile2="${tempDir}/state2.yaml"
printf '%s\n' "$stateYaml2" > "$tempYamlFile2"
if checkTruthy "$configValue_color"; then
    diff -u -U3 --color "$tempYamlFile1" "$tempYamlFile2"
else
    diff -u -U3 "$tempYamlFile1" "$tempYamlFile2"
fi
rm -R "$tempDir"
\end{minted}
\caption{Átmeneti fájlok kezelése utána}
\end{listing}

...

\subsection{TODO}

TODO

\section{A Git Chord webes felületének felépítése}

TODO

\cleardoublepage

\chapter{Összegzés és alkalmazások}

Hello.

\cleardoublepage

\appendix

\chapter{Teszteredmények}

Hello.

\cleardoublepage

\phantomsection
\addcontentsline{toc}{chapter}{\biblabel}
\printbibliography[title=\biblabel]
\cleardoublepage

\phantomsection
\addcontentsline{toc}{chapter}{\lstfigurelabel}
\listoffigures
\cleardoublepage

\phantomsection
\addcontentsline{toc}{chapter}{\lsttablelabel}
\listoftables
\cleardoublepage

\phantomsection
\addcontentsline{toc}{chapter}{\lstalgorithmlabel}
\listofalgorithms
\cleardoublepage

\phantomsection
\addcontentsline{toc}{chapter}{\lstcodelabel}
\lstlistoflistings
\cleardoublepage

\end{document}

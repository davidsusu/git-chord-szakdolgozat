\documentclass[final]{elteikthesis}[2025/03/25]

\usepackage{abstract}
\usepackage[newfloat]{minted}
\removefromtoclist[float]{lol}

\title{Git Chord: Git tárolók meta-állapotainak hatékony kezelése}
\date{2025}

\author{Horváth Dávid}
\degree{Programtervező Informatikus BSc}

\supervisor{Dr. Cserép Máté}
\affiliation{egyetemi adjunktus}

\university{Eötvös Loránd Tudományegyetem}
\faculty{Informatikai Kar}
\department{Programozáselmélet és Szoftvertechnológiai\\ Tanszék}
\city{Budapest}
\logo{elte_cimer_szines}

\addbibresource{references.bib}

\begin{document}

\documentlang{hungarian}

\listoftodos
\cleardoublepage

\maketitle

\renewcommand{\abstractname}{Témabejelentő}

\begin{abstract}
Szakdolgozatomban bemutatom a Git Chord programot és az ahhoz készített néhány további segédeszközt.
A Git Chord egy POSIX shellben írt kiterjesztés a Git
verziókezelő rendszerhez, amely konfigurálható módon lehetővé teszi egy helyi tároló elemeinek
(branchek, tagek, stb.)
aktuális állapotából összeálló konstelláció mentését és visszaállítását,
illetve ezeknek a mentett meta-állapotoknak a böngészését, törlését, migrálását, másokkal való megosztását.

A Git Chord kényelmesebb megoldást kínál a korábbi meta-állapotok visszaállítására,
mint a Git által beépítetten támogatott reflog.
A reflog továbbá csak a helyi tárolón használható,
míg a Git Chord mentett meta-állapotai könnyen megoszthatók,
például migráció, deployment, archiválás vagy akár segítségkérés céljából.

A mentett meta-állapotok technikailag a tárolón belüli
speciális brancheken elhelyezett, teljesen reguláris commitok,
melyek többedik parent commitokként is hivatkoznak a meta-állapot által rögzített, hordozandó commitokra.
Így minden szükséges függőség garantáltan szállításra kerül a meta-állapot commitjával együtt,
ugyanakkor a mentések pehelysúlyúak, valójában nem történik költséges duplikáció.

A parancssori program mellett a szakdolgozati munka része egy webes GUI,
valamint az előbbire épülő kiegészítők néhány elterjedt
általános asztali integrált fejlesztői környezethez (például VS Code, Eclipse).
Az egyes kiegészítők a közös főképernyő mellett
az adott IDE sajátosságaihoz igazított egyedi működéseket is tartalmaznak (context menük, akciók, stb.).
A programnak a szakdolgozati védés keretében történő bemutatása során
nagy részben a webes felületre támaszkodom majd.

A fejlesztői-üzemeltetői hétköznapok során történő felhasználási eseteken túl
kitérek a programnak a verziókezelő rendszerek oktatásában való lehetséges alkalmazására.
Ennek alapgondolata, hogy a hallgatók által
a saját helyi tárolójukon végzett lokális módosítások eredménye
az értékelő oktatóval könnyen megosztható a Git Chord által mentett meta-állapotként.
\end{abstract}

\tableofcontents
\cleardoublepage

\chapter{Bevezetés}

Egy \textit{verziókezelő rendszer} feladata, hogy valamilyen adott tartalom (például programkód)
folyamatos fejlesztését megkönnyítse, annak evolúcióját támogató technikai keretet biztosítson.
A verziókezelt tartalom újabb és újabb verziói egy időbeli sorba rendeződnek,
a verziók és a köztük lévő különbségek visszakövethetők,
az egyes állapotok szükség esetén visszaállíthatók.
Emellett a fejlesztés el is ágazhat,
ekkor az újabb verziók párhuzamosan futó \textit{branch}eken kerülnek tárolásra.
Az elágazó fejlesztési történet lineáris láncolat helyett tehát egy aciklikus gráf. 
A modern elosztott verziókezelőkben a branchek kezelésére igen szofisztikált eszköztár áll rendelkezésre;
az olyan műveletek, mint új branchek létrehozása, meglévők egyesítése,
váltás a branchek között,
vagy akár a történet visszamenőleges manipulációja
általában egyszerű parancsok végrehajtásával elvégezhetők.

Jelenleg a legelterjedtebb verziókezelő a nyílt forráskódú \textit{Git}.
Filozófiájához hozzátartozik,
hogy a teljes verziókezelési folyamatot a helyi gépen kell elvégezni,
ezt lehet majd szinkronizálni más gépekre, például egy központi szerverre.
Mind a Git szervert, mind a lokális másolatot tárolónak (repository),
vagy röviden \textit{repó}nak nevezzük.
A Git elosztott rendszer,
használatához nincs feltétlenül szükség központi tárolóra.
Egy módosításcsomag publikálása két lépésből áll.
Elsőként a \textit{commit} művelettel létrehozzuk az új tárolt állapotot (snapshot);
a bevett terminológia szerint ezt is commitnak nevezik.
[TODO: a commit tartalma]
Alapértelmezetten az aktuális branch mentett állapota egyúttal az újonnan létrejött commitra ugrik.
Második lépésben a \textit{push} művelettel publikálhatjuk az adott branch (vagy branchek) állapotát egy szerverre.
A két lépés egymástól független,
és gyakran számos több commit létrehozása is megtörténik lokálisan,
mire sor kerülne a push műveletre.
A brancheken kívül \textit{tag}ek is létrehozhatók,
ezek hasonlóak a branchekhez,
a fő különbség, hogy véglegesek:
ha egyszer létrehoztunk egy adott commitra mutató taget,
az a továbbiakban immutábilisen erre a commitra fog mutatni
(ha mégis mozgatni szeretnénk, törölni kell, majd újra létrehozni).
Valójában kétféle tag létezik,
[TODO: a kétféle tag]
Általában (bár ez megkerülhető) a fejlesztők a saját gépükön
egyszerre tárolják a teljes verziótörténetet,
beleértve a számukra érdekes brancheket, gyakran az összeset.
Amikor a \textit{clone} művelettel lemásolunk, leklónozunk egy repót,
alapértelmezetten az összes branch és tag letöltésre kerül a teljes történettel együtt.
[vagyis sok van, itt jön be a nehézség; kezdők...]
[meta-verziókezelés, bár egy elágazó gráf, mindig van egy pillanatnyi meta-állapot]

\cleardoublepage

\chapter{Felhasználói dokumentáció}

Hello.

\cleardoublepage

\chapter{Fejlesztői dokumentáció}

Hello.

\cleardoublepage

\chapter{Összegzés és alkalmazások}

Hello.

\cleardoublepage

\appendix

\chapter{Teszteredmények}

Hello.

\cleardoublepage

\phantomsection
\addcontentsline{toc}{chapter}{\biblabel}
\printbibliography[title=\biblabel]
\cleardoublepage

\phantomsection
\addcontentsline{toc}{chapter}{\lstfigurelabel}
\listoffigures
\cleardoublepage

\phantomsection
\addcontentsline{toc}{chapter}{\lsttablelabel}
\listoftables
\cleardoublepage

\phantomsection
\addcontentsline{toc}{chapter}{\lstalgorithmlabel}
\listofalgorithms
\cleardoublepage

\phantomsection
\addcontentsline{toc}{chapter}{\lstcodelabel}
\lstlistoflistings
\cleardoublepage

\end{document}
